\chapter{CP-Verletzung in B-Meson-Systemen}
\section{Diskrete Symmetrietransformationen}
Symmetrien sind in der Physik von zentraler Bedeutung. Gemäß dem Noether-Theorem existiert in der klassischen Physik zu jeder kontinuierlichen Symmetrie eine Erhaltungsgröße. In quantenmechanischen Systemen können wir drei diskrete Symmetrietransformationen betrachten:
\begin{enumerate}
\item \textbf{Parität $\mathcal{P}$:} \\
      Bei der Paritätsoperation wird das Vorzeichen der kartesischen Ortskoordinaten umgekehrt. Dies entspricht einer Punktspigelung.
\item \textbf{Ladungskonjugation $\mathcal{C}$:} \\
      Jedes Teilchen wird durch sein Antiteilchen ersetzt.
\item \textbf{Zeitumkehr $\mathcal{T}$:} \\
      Das Vorzeichen auf der Zeitachse wird umgekehrt. Da in der vorligenden Arbeit allerdings nur die CP-Verletzung gemessen werden soll, wird die Zeitumkehr im folgenden vernachlässigt.
\end{enumerate}
Entgegen der klassischen Intuition konnte Wu 1956 nachweisen, dass die Parität im $\beta$-Zerfall und damit in der schwachen Wechselwirkung nicht erhalten ist. Weitere Experimente zeigen, dass die schwache Wechselwirkung die Parität maximal verletzt: Neutrinos, die nur schwach wechselwirken können, sind stets \glqq linkshändig\grqq (Spin und Impuls antiparallel), Antineutrinos dagegen immer \glqq rechtshändig\grqq (Spin und Impuls parallel). Da der Spin im Gegensatz zum Impuls invariant unter $\mathcal{P}$-Transformation ist, würde diese Operation aus einem linkshändigen Neutrino ein rechtshändiges machen, was in der Nautr nicht realisiert ist.

Damit ist offensichtlich, dass die schwache Wechselwirkung auch die Ladungskonjugation verletzt: Wendet man die $\mathcal{C}$-Transformation auf ein linkshändiges Neutrino an, so erhält man ein linkshändiges Antineutrino. Dieses existiert aber wie bereits erwähnt nicht. Analog gilt die Überlegung auch für Antineutrinos.

\subsection{Scheinbare $\mathcal{CP}$-Invarianz}
Wendet man nun aber die Transformationen $\mathcal{P}$ und $\mathcal{C}$ direkt hintereinander an, so ergibt sich zunächst kein Widerspruch zur Natur (siehe Abb. \ref{fig:cp_invarianz}). Aus einen linkshändigen Neutrino wird ein rechtshändiges Antineutrino. Im Jahre 1964 wurde dann allerdings im Zerfall neutraler K-Mesonen erstmals $\mathcal{CP}$-Verletzung nachgewiesen. \cite{kleinknecht}

\begin{figure}[hptb]
\centering
\includegraphics[width = 0.8\textwidth]{cp_invarianz}
\caption{Scheinbare $\mathcal{CP}$-Invarianz: Während eine reine $\mathcal{P}$- oder $\mathcal{C}$-Transformation zu in der Natur nicht realisierten Zuständen führt, scheint es bei der kombinierten $\mathcal{CP}$-Transformation keinen Widerspruch zu geben (dünne Pfeile: Impulsausrichtung, dicke Pfeile: Spinausrichtung).}
\label{fig:cp_invarianz}
\end{figure}



\section{\CP-Verletzung in der Mischung}
Die Flavoureigenzustände $\Ket{B^0} = \Ket{\overline{b}d}$ und $\Ket{\overline{B^0}} = \Ket{b\overline{d}}$ entsprechen nicht den Masseneigenzuständen. Wir definieren daher die normierten Zustände
\begin{align}
\Ket{B_h} = p \Ket{B^0} - q \Ket{\overline{B^0}} \label{eq:b_heavy}\\ 
\Ket{B_l} = p \Ket{B^0} + q \Ket{\overline{B^0}} \label{eq:b_light}\\
\text{mit} \quad |p|^2 + |q|^2 = 1
\end{align}
welche eine definierte Masse und Zerfallsbreite besitzen. Sie sind auch Eigenzustände eines nicht-hermiteschen Hamiltonoperators (Nichthermitizität wegen des möglichen Zerfalls der Teilchens). Dieser setzt sich zusammen aus den hermiteschen Massenoperatoren $M$ und $\Gamma$. Notieren wir die lineare Superposition der Zustände \ref{eq:b_heavy} und \ref{eq:b_light} als $\begin{pmatrix} p \\ q \end{pmatrix}$, so nimmt die zeitabhängige Schrödingergleichung die Form
\begin{align}
\im \diff{}{t}\begin{pmatrix} p \\ q \end{pmatrix} = \left(M - \frac{\im}{2} \Gamma\right) \begin{pmatrix} p \\ q \end{pmatrix}
\end{align}
an und führt zur folgenden zeitlichen Entwicklung der Zustände:
\begin{align}
\nonumber \Ket{B_{h/l}(t)} &= \e^{-\im m_{h/l}t-\frac{1}{2}\Gamma_{h/l}t}\Ket{B_{h/l}(0)} \\
                           &= \e^{-\gamma_{h/l}t}(p\Ket{B^0} \mp q\Ket{\overline{B^0}}) \\
&\text{mit} \quad \gamma_{h/l} = \im m_{h/l}+\frac{\Gamma_{h/l}}{2}
\end{align}
Hierbei ist $\gamma_{h/l}$ so definiert, dass $-\im\gamma_{h/l} = m_{h/l}-\frac{\im}{2}\Gamma_{h/l}$ die Eigenwerte des Hamiltonoperators $\mathcal{H} := \left(M - \frac{\im}{2} \Gamma\right)$ sind. Umgeschrieben auf die Flavoureigenzustände erhält man:
\begin{align}
\nonumber \Ket{B^0(t)} &= \frac{1}{2p}\left(\Ket{B_h} + \Ket{B_l}\right) \\
                       &= \frac{1}{2}\left[ (\e^{-\gamma_h t}+\e^{-\gamma_l t})\Ket{B^0} - \frac{q}{p}(\e^{-\gamma_h t}-\e^{-\gamma_l t})\Ket{\overline{B^0}}\right] \label{eg:b(t)}
\end{align}
Die Wahrscheinlichkeit für den Übergang eines $\Ket{B^0}$ (zum Zeitpunkt $t=0$) in ein $\Ket{\overline{B^0}}$ beträgt:
\begin{align}
\nonumber P(B^0\rightarrow\overline{B^0})(t) &= |\Braket{\overline{B^0}|B^0(t)}|^2 \\
                                        &= \frac{1}{4} \left|\frac{q}{p}\right|^2 \left[\e^{-\Gamma_h t} + \e^{-\Gamma_l t} - 2\e^{-\frac{1}{2}(\Gamma_h + \Gamma_l) t}\cos(\Delta m_d t)\right] \\
&\text{mit} \quad \Delta m_d = m_h - m_l
\end{align}

Analog gilt für die Übergangswahrscheinlichkeit eines $\Ket{\overline{B^0}}$ in ein $\Ket{B^0}$:
\begin{align}
P(\overline{B^0}\rightarrow B^0)(t) = \frac{1}{4} \left|\frac{p}{q}\right|^2 \left[\e^{-\Gamma_h t} + \e^{-\Gamma_l t} - 2\e^{-\frac{1}{2}(\Gamma_h + \Gamma_l) t}\cos(\Delta m_d t)\right] 
\end{align}

Es kommt daher in der Mischung zur \CP-Verletzung, wenn die Oszillation ungleichmäßig verläuft, anders ausgedrückt:
\begin{align}
\text{\CP-Verletzung in der Mischung} \qquad \Longleftrightarrow \qquad \left|\frac{p}{q}\right| \neq 1 
\end{align}

\section{Direkte \CP-Verletzung}
Die Zerfallsamplituden der neutralen $B^0$-Mesonen in einen Endzustand $\Ket{f}$ bzw. seinen \CP-konjugierten Zustand $\Ket{\overline{f}}$ sind definiert als
\begin{alignat}{2}
\nonumber A_f &= \Braket{f|\mathcal{H}|B^0}, && \qquad A_{\overline{f}} = \Braket{\overline{f}|\mathcal{H}|B^0}, \\
          \overline{A_f} &= \Braket{f|\mathcal{H}|\overline{B^0}}, && \qquad  \overline{A_{\overline{f}}} = \Braket{\overline{f}|\mathcal{H}|\overline{B^0}}. \label{eq:decay_amplitudes}
\end{alignat}
Dabei bezeichnet $\mathcal{H}$ einen Hamiltonoperator der schwachen Wechselwirkung. Ist \CP erhalten, dann sollten die Zerfallsraten, ergo auch die Zerfallsamplituden eines $B^0$ nach $f$ sowie eines $\overline{B^0}$ nach $\overline{f}$ gleich sein. Dies bedeutet:
\begin{align}
\text{Direkte \CP-Verletzung} \qquad \Longleftrightarrow \qquad \frac{|A_f|}{|\overline{A_{\overline{f}}}|} \neq 1 \quad \text{bzw.} \quad \frac{|\overline{A_f}|}{|A_{\overline{f}}|} \neq 1
\end{align}


\section{\CP-Verletzung in der Interferenz}
Die Zustände \ref{eq:b_heavy} und \ref{eq:b_light} haben eine nahezu gleiche Anzahl an Zerfällskanäle. Dies hat zur Folge, dass die Lebensdauern des schweren und leichten Zustands innerhalb weniger Prozent gleich sind:
\begin{align}
\Gamma := \Gamma_h = \Gamma_l \label{eq:Gamma}
\end{align}

Weiterhin sagt das Standard Modell nur eine kleine \CP-Verletzung in der \Bd - \Bdbar - Mischung voraus, sodass
\begin{align}
\left|\frac{p}{q}\right| = 1 \qquad \text{in} \mathcal{O}(10^{-3}). \label{eg:pq_approx}
\end{align}

Für das B-Meson-System bleibt daher nur die Möglichkeit der \CP-Verletzung in der Interferenz von Mischung und direktem Zerfall. Der in dieser Arbeit betrachtete Zerfallskanal $B_d^0 \rightarrow J/\Psi K_s^0$ hat einen \CP-Eigenzustand als Endzustand (\CP $\Ket{\JPsi\Kshort} = -\Ket{\JPsi\Kshort}$). In Anlehnung an \ref{eq:decay_amplitudes} sind die Zerfallsamplituden hier definiert als
\begin{align}
\nonumber A_f := \Braket{f|B^0(t)}, \qquad \overline{A_{f}} := \Braket{f|\mathcal{H}|\overline{B^0}}
\end{align}

Mit Blick auf die Zerfallsamplituden der Masseneigenzustände wird die komplexe Größe
\begin{align}
\lambda := \frac{q\overline{A_f}}{pA_f} \label{eq:lambda}
\end{align}
definiert. Ausgehend von Gleichung \ref{eg:b(t)} sowie mit Hilfe fer Gleichungen (\ref{eq:Gamma}), (\ref{eg:pq_approx}) und (\ref{eq:lambda}) gilt für die Zerfallsrate eines anfänglich reinen \Bd-Zustands:
\begin{align}
\nonumber \Gamma (B^0 \rightarrow \JPsi\Kshort) &= \frac{1}{4}\left| (\e^{-\gamma_h t}+\e^{-\gamma_l t})A_f - \frac{q}{p}(\e^{-\gamma_h t}-\e^{-\gamma_l t})\overline{A_f}\right|^2 \\
&= \frac{1}{2} \left|A_f\right|^2\e^{-\Gamma t} \left[1+|\lambda|^2 + (1-|\lambda|^2)\cos(\Delta m_d t) - 2\mathrm{Im}(\lambda)\sin(\Delta m_d t)\right]
\end{align}
Analog:
\begin{align}
\Gamma (\overline{B^0} \rightarrow \JPsi\Kshort) &= \frac{1}{2} \left|A_f\right|^2\e^{-\Gamma t} \left[1+|\lambda|^2 -(1-|\lambda|^2)\cos(\Delta m_d t) + 2\mathrm{Im}(\lambda)\sin(\Delta m_d t)\right]
\end{align}

Damit kann die vom Standard Modell prognostizierte \CP-verletzende Asymmetrie 
\begin{align}
\mathcal{A}_{\text{\CP}} &= \frac{\Gamma (\overline{B^0} \rightarrow \JPsi\Kshort) - \Gamma (B^0 \rightarrow \JPsi\Kshort)}{\Gamma (\overline{B^0} \rightarrow \JPsi\Kshort) + \Gamma (B^0 \rightarrow \JPsi\Kshort)} \\
&= -\frac{1-|\lambda|^2}{1+|\lambda|^2}\cos(\Delta m_d t) + \frac{2\mathrm{Im}(\lambda)}{1+|\lambda|^2}\sin(\Delta m_d t) \\
&=: \CJPsi \cos(\Delta m_d t) + \SJPsi \sin(\Delta m_d t)
\end{align}
berechnet werden und vereinfacht sich - da $\Ket{\JPsi\Kshort}$ ein \CP-Eigenzustand ist, gilt $|\lambda| = 1$ - hier zu
\begin{align}
\mathcal{A}_{\text{\CP}} = \mathrm{Im}(\lambda)\sin(\Delta m_d t) .
\end{align}

Damit kann im B-Meson-System, insbesondere im Zerfall $B_d^0 \rightarrow J/\Psi K_s^0$ durch Messung der Asymmetrie-Amplitude $\SJPsi$ \CP-Verletzung in der Interferenz gemessen werden.

\begin{align}
\text{\CP-Verletzung in der Interferenz} \qquad \Longleftrightarrow \qquad \SJPsi = \mathrm{Im}(\lambda)\neq 0
\end{align}

\section{CKM-Formalismus}