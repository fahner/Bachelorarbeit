\documentclass[a4paper]{scrbook}
\usepackage[utf8]{inputenc}
\usepackage[ngerman]{babel}

\usepackage{array}            % Tabellen im Mathemodus (Matrizen)
\usepackage{graphicx}         % Einbindung von Grafiken
\usepackage{braket}           % ermöglicht die Verwendung von Diracs Bra-Ket-Notation
\usepackage{esdiff}           % für Ableitungen
\usepackage{amssymb}
\usepackage[squaren]{SIunits} % zur Darstellung physikalischer Einheiten (Option squaren verhindert Probleme mit amssymb Paket)
\usepackage{subfigure}        % mehrere Bilder in einer figure-Umgebung


\usepackage{hyperref}         % für automatische Verlinkungen in der pdf
\usepackage{datatool}	      % muss verwendet werden, da glossaries sonst einen Fehler verursacht
\usepackage[toc]{glossaries}  % Glossar

\graphicspath{{./Abbildungen/}}

\makeglossaries
\makeindex

% Glossar
\newglossaryentry{OST}{name={OST}, description={Opposite Side Tagger}}
\newglossaryentry{Toy MC}{name={Toy MC}, description={Zur Validierung des Fitters werden zufällig Daten gemäß einer gewünschten Verteilung generiert und im Anschluss gefittet}}


% Definiere Kürzel
\newcommand{\SJPsi}{S_{J/\Psi K_s^0}}
\newcommand{\CJPsi}{C_{J/\Psi K_s^0}}
\newcommand{\Bd}{$B_d^0$}
\newcommand{\Bdbar}{$\overline{B_d^0}$}
\newcommand{\CP}{$\mathcal{CP}$}
\newcommand{\im}{\mathrm{i}}
\newcommand{\e}{\mathrm{e}}
\newcommand{\ps}{\pico\second}

\newcommand{\JPsi}{J/\Psi}
\newcommand{\Kshort}{K_s^0}




\begin{document}

% Binde Titelseite ein
\begin{titlepage}
\thispagestyle{empty}
\begin{center}
 
\Large\textbf{Fakultät für Physik und Astronomie\\
Ruprecht-Karls-Universität Heidelberg}

%\vspace{15cm}
\vfill
\normalsize
Bachelorarbeit in Physik\\
eingereicht von\\
\vspace{0.5cm}
\Large\textbf{Patrick Fahner}\\
\normalsize
\vspace{0.5cm}
geboren in Mannheim (Deutschland)\\
\vspace{0.5cm}
\Large\textbf{August 2013}

\newpage
\thispagestyle{empty}
\cleardoublepage
\thispagestyle{empty}
\normalsize
\boldmath
\Huge{\textbf{Messung von $\sin(2\beta)$ im Zerfall \Decaychannel\ am LHCb-Experiment}}
\unboldmath
\vfill
\normalsize
Diese Bachelorarbeit wurde von Patrick Fahner am\\
Physikalischen Institut der Universität Heidelberg\\
unter der Aufsicht von\\
Prof. Dr. Stephanie Hansmann-Menzemer \\
durchgeführt.
\end{center}
\end{titlepage}
\newpage
\hrule
\section*{\abstractname}
Ziel dieser Arbeit ist die Bestimmung des CKM-Winkels $\sin(2\beta)$. Hierzu wird der Zerfallskanal \Decaychannel\ ausgewertet in Daten, die 2012 am LHCb-Detektor bei einer Schwerpunktsenergie von $\sqrt{s}=8\tera\electronvolt$ aufgenommen wurden und einer integrierten Luminosität von ungefähr $2\femto\barn^{-1}$ entsprechen. Als Basis dient eine LHCb-Analyse der im Jahre 2011 aufgenommenen Daten \cite{lhcb-paper}. Das Ergebnis 
\begin{align*}
\sin(2\beta) = 0,711 \pm 0,059(\text{stat.}) \pm 0,033(\text{syst.})
\end{align*}
ist sowohl mit dem Resultat der 2011-Analyse $\sin(2\beta) = 0,72 \pm 0,07 (\text{stat.}) \pm 0,04 (\text{syst.})$ \cite{lhcb-paper} als auch mit dem aktuellen Welt-Mittelwert von $\sin(2\beta) = 0,682 \pm 0,019$ \cite{pdg-average} kompatibel. Ein wichtiger Punkt dieser Arbeit ist die Abschätzung systematischer Effekte. Sie zeigt, dass die systematische Unsicherheit im Wesentlichen von der Kalibration der Algorithmen zur Bestimmung des Produktionsflavours bestimmt ist. \\
\hrule
\vfill
\hrule
\selectlanguage{english}
\hrule
\ZifferPunktAus
\section*{\abstractname}
This thesis provides a measurement of the CKM-angle $\sin(2\beta)$. It evaluates the decay channel \Decaychannel\ of data taken by the LHCb experiment at a center of mass energy of $\sqrt{s}=8\tera\electronvolt$ with an integrated luminosity of about $2\femto\barn^{-1}$. The same analysis strategy as an LHCb analysis of the data taken in 2011 \cite{lhcb-paper} was applied and results
\begin{align*}
\sin(2\beta) = 0.711 \pm 0.059(\text{stat.}) \pm 0.033(\text{syst.})
\end{align*}
which is in good agreement with the 2011 LHCb result $\sin(2\beta) = 0.72 \pm 0.07 (\text{stat.})$ $\pm 0.04 (\text{syst.})$ \cite{lhcb-paper} as well as the world average $\sin(2\beta) = 0.682 \pm 0.019$ \cite{pdg-average}. Furthermore this thesis focuses on the study of systematic effects. The main contribution to the systematic uncertainty is the calibration of the flavour tagging algorithms, which determine the B production flavour. \\ \hrule
\ZifferPunktAn

\selectlanguage{ngerman}




\tableofcontents
\chapter{Das LHCb-Experiment}

\chapter{CP-Verletzung in B-Meson-Systemen}
\section{Diskrete Symmetrietransformationen}
Symmetrien sind in der Physik von zentraler Bedeutung. Gemäß dem Noether-Theorem existiert in der klassischen Physik zu jeder kontinuierlichen Symmetrie eine Erhaltungsgröße. In quantenmechanischen Systemen können wir drei diskrete Symmetrietransformationen betrachten:
\begin{enumerate}
\item \textbf{Parität $\mathcal{P}$:} \\
      Bei der Paritätsoperation wird das Vorzeichen der kartesischen Ortskoordinaten umgekehrt. Dies entspricht einer Punktspigelung.
\item \textbf{Ladungskonjugation $\mathcal{C}$:} \\
      Jedes Teilchen wird durch sein Antiteilchen ersetzt.
\item \textbf{Zeitumkehr $\mathcal{T}$:} \\
      Das Vorzeichen auf der Zeitachse wird umgekehrt. Da in der vorligenden Arbeit allerdings nur die CP-Verletzung gemessen werden soll, wird die Zeitumkehr im folgenden vernachlässigt.
\end{enumerate}
Entgegen der klassischen Intuition konnte Wu 1956 nachweisen, dass die Parität im $\beta$-Zerfall und damit in der schwachen Wechselwirkung nicht erhalten ist. Weitere Experimente zeigen, dass die schwache Wechselwirkung die Parität maximal verletzt: Neutrinos, die nur schwach wechselwirken können, sind stets \glqq linkshändig\grqq (Spin und Impuls antiparallel), Antineutrinos dagegen immer \glqq rechtshändig\grqq (Spin und Impuls parallel). Da der Spin im Gegensatz zum Impuls invariant unter $\mathcal{P}$-Transformation ist, würde diese Operation aus einem linkshändigen Neutrino ein rechtshändiges machen, was in der Nautr nicht realisiert ist.

Damit ist offensichtlich, dass die schwache Wechselwirkung auch die Ladungskonjugation verletzt: Wendet man die $\mathcal{C}$-Transformation auf ein linkshändiges Neutrino an, so erhält man ein linkshändiges Antineutrino. Dieses existiert aber wie bereits erwähnt nicht. Analog gilt die Überlegung auch für Antineutrinos.

\subsection{Scheinbare $\mathcal{CP}$-Invarianz}
Wendet man nun aber die Transformationen $\mathcal{P}$ und $\mathcal{C}$ direkt hintereinander an, so ergibt sich zunächst kein Widerspruch zur Natur (siehe Abb. \ref{fig:cp_invarianz}). Aus einen linkshändigen Neutrino wird ein rechtshändiges Antineutrino. Im Jahre 1964 wurde dann allerdings im Zerfall neutraler K-Mesonen erstmals $\mathcal{CP}$-Verletzung nachgewiesen. \cite{kleinknecht}

\begin{figure}[hptb]
\centering
\includegraphics[width = 0.8\textwidth]{cp_invarianz}
\caption{Scheinbare $\mathcal{CP}$-Invarianz: Während eine reine $\mathcal{P}$- oder $\mathcal{C}$-Transformation zu in der Natur nicht realisierten Zuständen führt, scheint es bei der kombinierten $\mathcal{CP}$-Transformation keinen Widerspruch zu geben (dünne Pfeile: Impulsausrichtung, dicke Pfeile: Spinausrichtung).}
\label{fig:cp_invarianz}
\end{figure}



\section{\CP-Verletzung in der Mischung}
Die Flavoureigenzustände $\Ket{B^0} = \Ket{\overline{b}d}$ und $\Ket{\overline{B^0}} = \Ket{b\overline{d}}$ entsprechen nicht den Masseneigenzuständen. Wir definieren daher die normierten Zustände
\begin{align}
\Ket{B_h} = p \Ket{B^0} - q \Ket{\overline{B^0}} \label{eq:b_heavy}\\ 
\Ket{B_l} = p \Ket{B^0} + q \Ket{\overline{B^0}} \label{eq:b_light}\\
\text{mit} \quad |p|^2 + |q|^2 = 1
\end{align}
welche eine definierte Masse und Zerfallsbreite besitzen. Sie sind auch Eigenzustände eines nicht-hermiteschen Hamiltonoperators (Nichthermitizität wegen des möglichen Zerfalls der Teilchens). Dieser setzt sich zusammen aus den hermiteschen Massenoperatoren $M$ und $\Gamma$. Notieren wir die lineare Superposition der Zustände \ref{eq:b_heavy} und \ref{eq:b_light} als $\begin{pmatrix} p \\ q \end{pmatrix}$, so nimmt die zeitabhängige Schrödingergleichung die Form
\begin{align}
\im \diff{}{t}\begin{pmatrix} p \\ q \end{pmatrix} = \left(M - \frac{\im}{2} \Gamma\right) \begin{pmatrix} p \\ q \end{pmatrix}
\end{align}
an und führt zur folgenden zeitlichen Entwicklung der Zustände:
\begin{align}
\nonumber \Ket{B_{h/l}(t)} &= \e^{-\im m_{h/l}t-\frac{1}{2}\Gamma_{h/l}t}\Ket{B_{h/l}(0)} \\
                           &= \e^{-\gamma_{h/l}t}(p\Ket{B^0} \mp q\Ket{\overline{B^0}}) \\
&\text{mit} \quad \gamma_{h/l} = \im m_{h/l}+\frac{\Gamma_{h/l}}{2}
\end{align}
Hierbei ist $\gamma_{h/l}$ so definiert, dass $-\im\gamma_{h/l} = m_{h/l}-\frac{\im}{2}\Gamma_{h/l}$ die Eigenwerte des Hamiltonoperators $\mathcal{H} := \left(M - \frac{\im}{2} \Gamma\right)$ sind. Umgeschrieben auf die Flavoureigenzustände erhält man:
\begin{align}
\nonumber \Ket{B^0(t)} &= \frac{1}{2p}\left(\Ket{B_h} + \Ket{B_l}\right) \\
                       &= \frac{1}{2}\left[ (\e^{-\gamma_h t}+\e^{-\gamma_l t})\Ket{B^0} - \frac{q}{p}(\e^{-\gamma_h t}-\e^{-\gamma_l t})\Ket{\overline{B^0}}\right] \label{eg:b(t)}
\end{align}
Die Wahrscheinlichkeit für den Übergang eines $\Ket{B^0}$ (zum Zeitpunkt $t=0$) in ein $\Ket{\overline{B^0}}$ beträgt:
\begin{align}
\nonumber P(B^0\rightarrow\overline{B^0})(t) &= |\Braket{\overline{B^0}|B^0(t)}|^2 \\
                                        &= \frac{1}{4} \left|\frac{q}{p}\right|^2 \left[\e^{-\Gamma_h t} + \e^{-\Gamma_l t} - 2\e^{-\frac{1}{2}(\Gamma_h + \Gamma_l) t}\cos(\Delta m_d t)\right] \\
&\text{mit} \quad \Delta m_d = m_h - m_l
\end{align}

Analog gilt für die Übergangswahrscheinlichkeit eines $\Ket{\overline{B^0}}$ in ein $\Ket{B^0}$:
\begin{align}
P(\overline{B^0}\rightarrow B^0)(t) = \frac{1}{4} \left|\frac{p}{q}\right|^2 \left[\e^{-\Gamma_h t} + \e^{-\Gamma_l t} - 2\e^{-\frac{1}{2}(\Gamma_h + \Gamma_l) t}\cos(\Delta m_d t)\right] 
\end{align}

Es kommt daher in der Mischung zur \CP-Verletzung, wenn die Oszillation ungleichmäßig verläuft, anders ausgedrückt:
\begin{align}
\text{\CP-Verletzung in der Mischung} \qquad \Longleftrightarrow \qquad \left|\frac{p}{q}\right| \neq 1 
\end{align}

\section{Direkte \CP-Verletzung}
Die Zerfallsamplituden der neutralen $B^0$-Mesonen in einen Endzustand $\Ket{f}$ bzw. seinen \CP-konjugierten Zustand $\Ket{\overline{f}}$ sind definiert als
\begin{alignat}{2}
\nonumber A_f &= \Braket{f|\mathcal{H}|B^0}, && \qquad A_{\overline{f}} = \Braket{\overline{f}|\mathcal{H}|B^0}, \\
          \overline{A_f} &= \Braket{f|\mathcal{H}|\overline{B^0}}, && \qquad  \overline{A_{\overline{f}}} = \Braket{\overline{f}|\mathcal{H}|\overline{B^0}}. \label{eq:decay_amplitudes}
\end{alignat}
Dabei bezeichnet $\mathcal{H}$ einen Hamiltonoperator der schwachen Wechselwirkung. Ist \CP erhalten, dann sollten die Zerfallsraten, ergo auch die Zerfallsamplituden eines $B^0$ nach $f$ sowie eines $\overline{B^0}$ nach $\overline{f}$ gleich sein. Dies bedeutet:
\begin{align}
\text{Direkte \CP-Verletzung} \qquad \Longleftrightarrow \qquad \frac{|A_f|}{|\overline{A_{\overline{f}}}|} \neq 1 \quad \text{bzw.} \quad \frac{|\overline{A_f}|}{|A_{\overline{f}}|} \neq 1
\end{align}


\section{\CP-Verletzung in der Interferenz}
Die Zustände \ref{eq:b_heavy} und \ref{eq:b_light} haben eine nahezu gleiche Anzahl an Zerfällskanäle. Dies hat zur Folge, dass die Lebensdauern des schweren und leichten Zustands innerhalb weniger Prozent gleich sind:
\begin{align}
\Gamma := \Gamma_h = \Gamma_l \label{eq:Gamma}
\end{align}

Weiterhin sagt das Standard Modell nur eine kleine \CP-Verletzung in der \Bd - \Bdbar - Mischung voraus, sodass
\begin{align}
\left|\frac{p}{q}\right| = 1 \qquad \text{in} \mathcal{O}(10^{-3}). \label{eg:pq_approx}
\end{align}

Für das B-Meson-System bleibt daher nur die Möglichkeit der \CP-Verletzung in der Interferenz von Mischung und direktem Zerfall. Der in dieser Arbeit betrachtete Zerfallskanal $B_d^0 \rightarrow J/\Psi K_s^0$ hat einen \CP-Eigenzustand als Endzustand (\CP $\Ket{\JPsi\Kshort} = -\Ket{\JPsi\Kshort}$). In Anlehnung an \ref{eq:decay_amplitudes} sind die Zerfallsamplituden hier definiert als
\begin{align}
\nonumber A_f := \Braket{f|B^0(t)}, \qquad \overline{A_{f}} := \Braket{f|\mathcal{H}|\overline{B^0}}
\end{align}

Mit Blick auf die Zerfallsamplituden der Masseneigenzustände wird die komplexe Größe
\begin{align}
\lambda := \frac{q\overline{A_f}}{pA_f} \label{eq:lambda}
\end{align}
definiert. Ausgehend von Gleichung \ref{eg:b(t)} sowie mit Hilfe fer Gleichungen (\ref{eq:Gamma}), (\ref{eg:pq_approx}) und (\ref{eq:lambda}) gilt für die Zerfallsrate eines anfänglich reinen \Bd-Zustands:
\begin{align}
\nonumber \Gamma (B^0 \rightarrow \JPsi\Kshort) &= \frac{1}{4}\left| (\e^{-\gamma_h t}+\e^{-\gamma_l t})A_f - \frac{q}{p}(\e^{-\gamma_h t}-\e^{-\gamma_l t})\overline{A_f}\right|^2 \\
&= \frac{1}{2} \left|A_f\right|^2\e^{-\Gamma t} \left[1+|\lambda|^2 + (1-|\lambda|^2)\cos(\Delta m_d t) - 2\mathrm{Im}(\lambda)\sin(\Delta m_d t)\right]
\end{align}
Analog:
\begin{align}
\Gamma (\overline{B^0} \rightarrow \JPsi\Kshort) &= \frac{1}{2} \left|A_f\right|^2\e^{-\Gamma t} \left[1+|\lambda|^2 -(1-|\lambda|^2)\cos(\Delta m_d t) + 2\mathrm{Im}(\lambda)\sin(\Delta m_d t)\right]
\end{align}

Damit kann die vom Standard Modell prognostizierte \CP-verletzende Asymmetrie 
\begin{align}
\mathcal{A}_{\text{\CP}} &= \frac{\Gamma (\overline{B^0} \rightarrow \JPsi\Kshort) - \Gamma (B^0 \rightarrow \JPsi\Kshort)}{\Gamma (\overline{B^0} \rightarrow \JPsi\Kshort) + \Gamma (B^0 \rightarrow \JPsi\Kshort)} \\
&= -\frac{1-|\lambda|^2}{1+|\lambda|^2}\cos(\Delta m_d t) + \frac{2\mathrm{Im}(\lambda)}{1+|\lambda|^2}\sin(\Delta m_d t) \\
&=: \CJPsi \cos(\Delta m_d t) + \SJPsi \sin(\Delta m_d t)
\end{align}
berechnet werden und vereinfacht sich - da $\Ket{\JPsi\Kshort}$ ein \CP-Eigenzustand ist, gilt $|\lambda| = 1$ - hier zu
\begin{align}
\mathcal{A}_{\text{\CP}} = \mathrm{Im}(\lambda)\sin(\Delta m_d t) .
\end{align}

Damit kann im B-Meson-System, insbesondere im Zerfall $B_d^0 \rightarrow J/\Psi K_s^0$ durch Messung der Asymmetrie-Amplitude $\SJPsi$ \CP-Verletzung in der Interferenz gemessen werden.

\begin{align}
\text{\CP-Verletzung in der Interferenz} \qquad \Longleftrightarrow \qquad \SJPsi = \mathrm{Im}(\lambda)\neq 0
\end{align}

\section{CKM-Formalismus}

\chapter{Datenselektion}

\chapter{Analyse / Fit}
\section{Fitmethode SFit}
\section{Wahrscheinlichkeitsdichtefunktion}
\begin{equation}
xxx     \label{eg:fit_pdf}
\end{equation}
\section{Fitergebnis} \label{kap:fitergebnis}
Wir erhalten schließlich:
\begin{align}
\SJPsi = xxx \pm xxx     \label{eq:fit_result}
\end{align}

\chapter{Abschätzung systematischer Unsicherheiten} \label{kap:systematik}
Der Fitter liefert zwar eine statistische Unsicherheit auf $\SJPsi$, allerdings ist eine Betrachtung der Systematik unerlässlich. Im Folgenden wird daher der Einfluss einiger Effekte auf das Fitergebnis untersucht und anschließend der systematische Fehler abgeschätzt.

\section{Fitmethode} \label{kap:fit_bias}
Die hier verwendete Maximum-Likelihood-Methode hat zwar den schönen Vorteil, dass das Fitergebnis nicht von der Einteilung des Histogramms abhängt, es ist jedoch nicht von vornherein ausgeschlossen, dass sie das Ergebnis verfälscht (ein sog. Bias produziert). Daher wird eine Toy Monte Carlo - Studie (kurz: Toy MC) durchgeführt. Dabei werden Daten zufällig nach einer Verteilung mit den gewünschten Parametern generiert und im Anschluss gefittet. Zur Generation der Massen- und Eigenzeitverteilung werden die aus den Fits erhaltenen Parameter verwendet (siehe Tabellen \ref{tab:fit_masse} und \ref{tab:fit_results}). Die einzige Ausnahme bildet $\SJPsi$, da diese zum Zeitpunkt dieser Studie noch verdeckt war. Hier wurde mit $\SJPsi = 0,72$ generiert. Entsprechend der Statistik im verwendeten Datensatz werden hier jeweils 20000 Ereignisse generiert. Durch mehrmaliges Wiederholen von Generation und Fit sollten die gefitteten Parameter am Ende gaußverteilt um die in der Generation verwendeten Parameter sein. Kommt es zu Abweichungen davon, so ist dies auf die Fitmethode oder eine fehlerhafte Implementation des Experimentators zurückzuführen. Um statistisch zuverlässige Aussagen treffen zu können, wurden in dieser Toy MC - Studie insgesamt 20000 Wiederholungen durchgeführt.

\begin{figure}[hptb]
\centering
\includegraphics[width = \textwidth]{fit_bias}
\caption{Verteilung der aus der Toy MC Studie erhaltenen Amplituden $\SJPsi$ (links) sowie die dazugehörigen Pulls (rechts)}
\label{fig:fit_bias}
\end{figure}

Abbildung \ref{fig:fit_bias} zeigt sowohl die Verteilung der gefitteten Amplitude $\SJPsi$ als auch die Pulls, die sich mittels $(\SJPsi^{\text{gefittet}} - \SJPsi^{\text{generiert}})/\sigma^{\text{gefittet}}$ berechnen lassen. Der Mittelwert der Amplitudenverteilung (links) $\SJPsi^{\text{ToyMC}} = 0,7234 \pm 0,0004$ weicht signifikant vom generierten Wert $\SJPsi = 0,72$ ab, es gibt also ein Bias. An der Pull-Verteilung lassen sich prinzipiell zwei Dinge beobachten:
\begin{enumerate}
    \item An der Verschiebung des Pull-Mittelwertes $\mu = 0,0522 \pm 0,0067$ von der Null sieht man deutlich, dass es - wie bereits erwähnt - einen kleinen, aber signifikanten Bias gibt. Indem dieser Bias mit der statistischen Unsicherheit aus dem Fitergebnis (siehe Gl. \ref{eq:fit_result}) multipliziert wird, erhält man eine Abschätzung der aus der Fitmethode resultierenden Unsicherheit:
        \begin{align}
        \delta\SJPsi^{Fit} = 0,0522 \cdot 0,0626 = 0,0033
        \end{align}

    \item Mit einem $\sigma = 0,941 \pm 0,005$ ist die Pull-Verteilung signifikant zu schmal. Bei einer zufälligen Streuung der Werte wird nämlich eigentlich ein $\sigma=1$ erwartet. Dies bedeutet, dass der Fit den statistischen Fehler überschätzt. Dieses Ergebnis kann später als Faktor zur Korrektur des statistischen Fehlers verwendet werden. Dieses Problem tritt auf, sobald man in den Toys Untergrund miteinbezieht (ohne Untergrund erhält man ein $\sigma=1,007\pm 0,005$ und ist damit kompatibel zur Eins). Es ist bekannt, dass die verwendete sFit-Methode die Fehlerpropagation (gerade bei Untergrund) nicht korrekt ausführt. Es wurde vom Betreuer eine Fehlerkorrektur implementiert, dabei handelt es sich jedoch um eine Näherung. Für eine tiefergehende Studie müsste die Fehlerkorrektur entsprechend analysiert werden.
\end{enumerate}

\subsubsection{Ursachen des Bias}
Es bleibt zu klären, welche Ursachen zu dem Bias führen. Es liegt in der Natur der Likelihood-Methode, dass für eine große Zahl $N$ an Messwerten die Abschätzung der Parameter gegen deren wahren Wert konvergiert. Demnach liegt die Vermutung nahe, dass zu wenig Ereignisse (\glqq Statistik\grqq) vorhanden ist. Daher wurden weitere Toy MC Studien mit unterschiedlicher Anzahl an Teilchen pro Toy durchgeführt. Die Ergebnisse sind in Tabelle \ref{tab:fit_bias_events} aufgeführt und in Abbildung \ref{fig:fit_bias_events} nochmals visualisiert. Man sieht, dass das Bias mit erhöhter Statistik deutlich reduziert wird und damit zu wenig Statistik als Hauptursache hierfür angesehen werden kann.

\begin{table}[hptb]
\centering
\caption{Toy MC Studien mit unterschiedlicher Anzahl an generierten Ereignissen pro Toy. Genannt wird der Mittelwert $\mu$ der $\SJPsi$-Pull-Verteilung}
\label{tab:fit_bias_events}
\begin{tabular}{cr@{$\pm$}l}
\hline \hline 
Teilchen pro Toy & \multicolumn{2}{c}{$\mu$}  \\ \hline
20000            &  0,0522 & 0,0067 \\
50000            &  0,0358 & 0,0067 \\
100000           &  0,0257 & 0,0068 \\
200000           &  0,0145 & 0,0068 \\ 
\hline \hline
\end{tabular}
\end{table}

\begin{figure}[hptb]
\centering
\includegraphics[width=\textwidth]{fit_bias_statistics}
\caption{Toy MC Studien mit unterschiedlicher Anzahl an generierten Ereignissen pro Toy. Als Maß für das Fit Bias dient der Mittelwert $\mu$ der $\SJPsi$-Pull-Verteilung}
\label{fig:fit_bias_events}
\end{figure}

\section{Kalibration der Flavour-Tagging-Algorithmen}
Im Fit wird bei den Parametern der Tagging Kalibration durch Einschränkung der Parameter deren statistische Fehler berücksichtigt. Bislang unbeachtet blieben die systematischen Fehler von $p_0$ und $p_1$, deren Einfluss im Folgenden untersucht wird. Leider sind die Studien zur Systematik des in dieser Analyse verwendeten Flavour-Taggings noch nicht abgeschlossen, sodass die systematischen Fehler $\delta p_0^{\text{stat.}}$ sowie $\delta p_1^{\text{stat.}}$ noch nicht vorliegen. Um dennoch ein Gefühl für den Einfluss des Flavour-Taggings zu bekommen, werden die systematischen Fehler der 2011-Analyse (\cite{lhcb-paper}) herangezogen. Die Annahme und Erwartung ist, dass sich die Systematiken zwischen 2011 und 2012 kaum unterscheiden. Für eine endgültige Analyse, muss dieser Schritt jedoch wiederholt werden, sobald die Analyse des Flavour-Taggings aus 2012 abgeschlossen ist. In dieser Arbeit werden nun folgende Werte und Fehler der Kalibrationsparameter $p_0$ und $p_1$ verwendet:
\begin{align}
p_0 &= 0,382 \pm 0,003\ \text{(stat.)} \pm 0,008\ \text{(syst.)}, \\
p_1 &= 0,981 \pm 0,024\ \text{(stat.)} \pm 0,012\ \text{(syst.)}.
\end{align}

\subsubsection{Variation der Parameter in den Daten}
Einen ersten Überblick über die Systematik erhält man, indem man
im regulären Eigenzeitfit die Startwerte der Parameter $p_0$ und $p_1$ um ihre systematischen Fehler variiert. In allen vier möglichen Kombinationen wird der systematische Fehler auf $p_0$ und $p_1$ addiert bzw. subtrahiert, dann der Fit durchgeführt und schließlich die Abweichung vom regulären Fitergebnis für $\SJPsi$ berechnet. Der (verdeckte) Referenzwert aus dem Fit beträgt
\begin{align}
\SJPsi = 0,5347 \pm 0,0626 
\end{align}

\begin{table}[hptb]
\centering
\caption{Variation des Fitergebnisses für $\SJPsi$ bei Veränderung der Startwerte für $p_0$ und $p_1$ $\pm$ ihrer systematischen Unsicherheiten}
\label{tab:syst_fit_calib_data}
\begin{tabular}{cc|r@{$\pm$}l|r}
\hline\hline
$p_0$  &  $p_1$  &  \multicolumn{2}{c|}{$\SJPsi$}  & $\Delta\SJPsi$   \\ \hline
0,382 - 0,008  &  0,981 - 0,024  &  0,5109 & 0,0604  &  -0,0238 \\
0,382 - 0,008  &  0,981 + 0,024  &  0,5103 & 0,0604  &  -0,0244 \\
0,382 + 0,008  &  0,981 - 0,024  &  0,5599 & 0,0649  &   0,0252 \\
0,382 + 0,008  &  0,981 + 0,024  &  0,5591 & 0,0648  &   0,0244 \\
\hline\hline
\end{tabular}
\end{table}

Die Ergebnisse sind Tabelle \ref{tab:syst_fit_calib_data} zu entnehmen. Die größte Abweichung beträgt hier $\Delta\SJPsi = 0,0252$.

\paragraph{Variation der Parameter in Toy MC}
Eine weitere Möglichkeit der Abschätzung besteht darin, sich entsprechende Toys zu generieren und diese dann zu fitten. Im Folgenden werden bei der Generation der Toys die Parameter $p_0$ und $p_1$ um ihre systematischen Unsicherheiten variiert, der Fit dann allerdings mit den ursprünglichen Parameterwerten durchgeführt. Als Referenzwert dient die aus dem Fit Bias (siehe Kapitel \ref{kap:fit_bias}) erhaltene Amplitude, da dort mit den regulären Parametern $p_0$ und $p_1$ generiert und gefittet wurde:
\begin{align}
\SJPsi = 0,7234 \pm 0,0004
\end{align}

\begin{table}[hptb]
\centering
\caption{Variation des Fitergebnisses für $\SJPsi$ bei Veränderung der Parameterwerte $p_0$ und $p_1$ $\pm$ ihrer systematischen Unsicherheiten bei der Generation von Toys}
\label{tab:syst_fit_calib_toys}
\begin{tabular}{cc|r@{$\pm$}l|r}
\hline\hline
$p_0$  &  $p_1$  &  \multicolumn{2}{c|}{$\SJPsi$}  & $\Delta\SJPsi$   \\ \hline
0,382 - 0,008  &  0,981 - 0,024  &  0,7515 & 0,0004  &   0,0281 \\
0,382 - 0,008  &  0,981 + 0,024  &  0,7565 & 0,0004  &   0,0331 \\
0,382 + 0,008  &  0,981 - 0,024  &  0,6909 & 0,0004  &  -0,0325 \\
0,382 + 0,008  &  0,981 + 0,024  &  0,6966 & 0,0004  &  -0,0244 \\
\hline\hline
\end{tabular}
\end{table}

Die Ergebnisse sind Tabelle \ref{tab:syst_fit_calib_toys} zu entnehmen, die dazugehörigen Plots werden in Abbildung \ref{fig:toys_tag_calib} gezeigt. Die größte Abweichung beträgt hier $\Delta\SJPsi = 0,0331$ und ist auch größer als bei Variation der Parameter in den Daten. Dementsprechend wird der systematische Fehler durch die Tagging Kalibrierung mit ebendiesem Wert abgeschätzt:

\begin{align}
\delta\SJPsi^{\text{FTK}} = 0,0331\ .
\end{align}

\begin{figure}[hptb]
\centering
\subfigure[{$p_0 - \Delta p_0$,  $p_1 - \Delta p_1$}]{\includegraphics[width=0.7\textwidth]{tagging_calibration--}}
\subfigure[{$p_0 + \Delta p_0$,  $p_1 - \Delta p_1$}]{\includegraphics[width=0.7\textwidth]{tagging_calibration+-}}
\subfigure[{$p_0 - \Delta p_0$,  $p_1 + \Delta p_1$}]{\includegraphics[width=0.7\textwidth]{tagging_calibration-+}}
\subfigure[{$p_0 + \Delta p_0$,  $p_1 + \Delta p_1$}]{\includegraphics[width=0.7\textwidth]{tagging_calibration++}}
\caption{Toy MC Studie zur Abschätzung der Systematik durch die Tagging Kalibration. Bei der Generation wurden die Taggingparameter $p_0=0,382$ und $p_1=0,981$ um ihre systematischen Unsicherheiten $\Delta p_0 = 0,008$ bzw. $\Delta p_1 = 0,024$ variiert, der Fit wurde dann mit den ursprünglichen Werten $p_0$ und $p_1$ durchgeführt.}
\label{fig:toys_tag_calib}
\end{figure}


\section{Einfluss einer zeitabhängigen Akzeptanz} \label{kap:akzeptanz}
In der Analyse wurde der Einfluss einer zeitabhängigen Detektorakzeptanz vernachlässigt. Nimmt man an, dass sich die Akzeptanz von \Bd- und \Bdbar-Mesonen nicht unterscheiden, so hat die Akzeptanz keinen Einfluss auf die \CP-Asymmetrie nach Gleichung \ref{eq:cp_asymm}, da sie sich hier herauskürzt. Beim Fit der Amplitude nach Gleichung \ref{eq:fit_pdf} ist dies aber nicht zwangsläufig so. Um die Vernachlässigung einer zeitabhängigen Akzeptanz zu rechtfertigen, wird zunächst eine Bestimmung der Akzeptanz durchgeführt und anschließend mit einer Toy MC Studie ihr Einfluss überprüft.

\subsubsection{Bestimmung einer Akzeptanzfunktion} 
\Bd-Mesonen haben eine relativ lange Lebensdauer. Um sie von kurzlebigem Untergrund zu unterscheiden, befinden sich auf den Triggern und dem Stripping entsprechende Cuts auf die Flugzeiten. Dies hat zur Folge, dass für kleine Flugzeiten ($ct \lesssim 0,3\pico\second$) kaum \Bd-Mesonen im Detektor registriert werden und es zu einem sog. \glqq Turn-On-Effekt\grqq kommt. Es hat sich herausgestellt (\cite{lhcb-paper}), dass dieser gut durch die Funktion
\begin{align}
\epsilon_1(t) = \frac{2}{\pi}\arctan[t\cdot \exp(at+b)]
\end{align}
parametrisiert wird.

Je länger ein \Bd-Meson lebt, desto schwieriger wird es, die Zerfallsprodukte im Detektor auf Grund seiner Geometrie nachzuweisen. Daher nimmt die Akzeptanz zu großen Zeiten hin wieder ab. Zur Parametrisierung fällt die Wahl auf eine lineare Funktion
\begin{align}
\epsilon_2(t) = 1 + \beta t \qquad (\beta < 0).
\end{align}
Die entsprechende gesamte Akzeptanzfunktion lautet demnach:
\begin{align}
\epsilon(t) = \epsilon_1(t) \cdot \epsilon_2(t) = \frac{2}{\pi}\arctan[t\cdot \exp(at+b)](1 + \beta t)
\end{align}
Zur Bestimmung der Parameter wird die Trennung von \Bd- und \Bdbar-Mesonen aufgehoben, sodass lediglich ein exponentieller Zerfall zu beobachten ist. Des weiteren wird die Selektion der Lebensdauer bei $0,3\pico\second$ nicht angewandt, sodass der Turn-On-Effekt auch wirklich sichtbar wird. Die Wahrscheinlichkeitsdichtefunktion für den Fit lautet somit:
\begin{align}
\mathcal{P}_{acc}(t) \propto \epsilon(t)\cdot \e^{-t/\tau} = \e^{-t/\tau}\cdot\frac{2}{\pi}\arctan[t\cdot \exp(at+b)](1 + \beta t)
\end{align}
Die beiden Parameter $\tau$ und $\beta$ sind stark miteinander korreliert. Für eine geeignete Bestimmung der Parameter der Akzeptanz-Funktion wird daher die Lebensdauer auf den PDG-Wert $\tau = 1,519 \pm 0,007 \pico\second$ \cite{pdg-tau} eingeschränkt, die anderen Parameter fließen. Die Ergebnisse sind in Tabelle \ref{tab:fit_akzeptanz} aufgeführt, die entsprechenden Plots in Abbildung \ref{fig:fit_akzeptanz}. 

\begin{table}[hptb]
\centering
\caption{Ergebnis des Fits zur Bestimmung der zeitlichen Akzeptanz}
\label{tab:fit_akzeptanz}
\begin{tabular}{lr@{$\pm$}ll}
\hline \hline 
Parameter & \multicolumn{2}{c}{Ergebnis} & \\ \hline
$\tau$    &  (1,519   & 0,007) & $\ps$ \\
$a $       &  (47,9    & 5,6) & $\ps^{-1}$ \\
$b$       &  -8,4    & 1,1 \\
$\beta$   &  (-0,0090 & 0,0076) & $\ps^{-1}$ \\ 
\hline \hline
\end{tabular}
\end{table}

\begin{figure}[hptb]
\centering
\includegraphics[width=\textwidth]{time_acceptance_fit}
\caption{Fit an die Eigenzeit-Verteilung aller \Bd-Mesonen mit eingeschlossener Akzeptanzfunktion (oben) sowie die entsprechende Pull-Verteilung (unten). Links: kurzlebiger Zeitbereich ($t<1,5\pico\second$), Rechts: gesamtes Flugzeitspektrum ($0\ps < t < 14\ps$)}
\label{fig:fit_akzeptanz}
\end{figure}

\subsubsection{Bestimmung des Einflusses}
Durch die Selektion der Lebensdauer ab $t = 0,3\ps$ in der Datenselektion spielt der Turn-On-Effekt im hier verwendeten Datensatz eigentlich keine Rolle. Dies wird dadurch deutlich, dass die Akzeptanzfunktion $\epsilon(0,3\ps) = 0,992$ und damit fast Eins ist. Auch am Ende des Analysebereichs beträgt die Akzeptanz noch $\epsilon(14\ps) = 0,874$. Daher liegt die Vermutung nahe, dass sich die Akzeptanz nicht gravierend auf das Fitergebnis auswirkt. Mit den oben bestimmten Parametern wird die zeitliche Akzeptanz bei der Erzeugung weiterer Toys berücksichtigt, der anschließende Fit aber ohne Akzeptanzfunktion durchgeführt. Die zur Erzeugung verwendeten Parameter entsprechen ansonsten denen in Kapitel \ref{kap:fit_bias}.

\begin{figure}[hptb]
\centering
\includegraphics[width = \textwidth]{time_acceptance_toys}
\caption{Untersuchung des Einflusses einer zeitlichen Akzeptanz: Verteilung der aus der Toy MC Studie erhaltenen Amplituden $\SJPsi$ (links) sowie die dazugehörigen Pulls (rechts)}
\label{fig:toys_acceptance}
\end{figure}

Der Mittelwert der Amplitudenverteilung beträgt $\SJPsi = 0,72420 \pm 0,00041$. Die Abweichung vom Referenzwert (Fit Bias, siehe Kap. \ref{kap:fit_bias} $\SJPsi = 0,72343 \pm 0,00041$,
\begin{align}
\delta\SJPsi^{\text{Akz.}} = 0,00077\,
\end{align}
wird zur Abschätzung des Fehlers durch Vernachlässigung einer Akzeptanzfunktion verwendet. Gerade im Vergleich zum Einfluss des Flavour Taggings auf $\SJPsi$ ist dieser hier sehr gering. Somit erscheint das Vorgehen, keine Akzeptanz im Eigenzeitfit zu verwenden, gerechtfertigt.


\section{Korrelation zwischen Masse und Eigenzeit}
Die sFit-Methode funktioniert dann gut, wenn der Untergrund der Massenverteilung eben ist und die Massenverteilung des Signals unabhängig von der gemessenen Eigenzeit ist. Es soll nun eine etwaige Korrelation zwischen Masse und Eigenzeit untersucht und der Einfluss auf $\SJPsi$ festgestellt werden. Dazu wird die Massenverteilung in vier verschiedenen Zeitbereichen gefittet, die Tabelle \ref{tab:mass_ct} zu entnehmen sind. Anschließend wird die gesamte Eigenzeitverteilung gefittet, dabei werden aber die Massenparameter des Signals auf die in den 4 Massenfits erhaltenen Werte fixiert. Die Ergebnisse des jeweiligen Fits sind ebenfalls in Tabelle \ref{tab:mass_ct} aufgeführt.

\begin{table}[hptb]
\centering
\caption{Einteilung der Eigenzeitbereiche sowie Fitresultate für $\SJPsi$ bei Fixierung der Masse auf die in den Zeitbereichen enthaltene Massenform. Weiterhin werden die Abweichung $\Delta\SJPsi$ vom regulären Datenfit und der Signalanzahl $N_{sig}$ eines jeden Eigenzeitbereichs genannt.}
\label{tab:mass_ct}
\begin{tabular}{c l r@{$\pm$}l c c}
\hline \hline
Nr. & Eigenzeitfenster des Massenfits & \multicolumn{2}{c}{$\SJPsi$} & $\Delta\SJPsi$ & $N_{sig}$\\ \hline
1 & $t \in [0.3, 0.7] \pico\second$ & 0,5318 & 0,0626 & -0,0029 & 2882 \\
2 & $t \in [0.7, 1.5] \pico\second$ & 0,5361 & 0,0625 & 0,0014 & 4066 \\
3 & $t \in [1.5, 3] \pico\second$ & 0,5361 & 0,0625 & 0,0014 & 4230 \\
4 & $t \in [3, 14] \pico\second$ & 0,5353 & 0,0624 & 0,0006 & 2177 \\ \hline \hline
\end{tabular}
\end{table}

Zur Abschätzung des Fehlers werden zunächst die Abweichungen $\Delta\SJPsi$ vom (noch verdeckten) Referenzwert aus dem regulären Eigenzeitfit $\SJPsi = 0,5347 \pm 0,0626$ berechnet (siehe Kap. \ref{kap:fitergebnis} und diese dann - gewichtet nach der Signalzahl $N_{sig}$ eines jeden Bereichs - quadratisch gemittelt:
\begin{align}
\delta\SJPsi^{m/t} = \sqrt{\frac{\sum N_i (\Delta\SJPsi)_i}{\sum N_i}} = 0,0018
\end{align}

\section{Eigenzeitauflösung} \label{kap:aufloesung}
Bei einer effektiven Eigenzeitauflösung von $\sigma_{eff} = (0,0665 \pm 0,0013)\ps$ im Vergleich zur \Bd-Oszillationsfrequenz $\delta m_d = (0,521 \pm 0.039)\hbar\ps$ erwartet man keine nennenswerten Effekte auf die Amplitude $\SJPsi$. Um überhaupt einen Effekt zu sehen, werden die Auflösungsparameter $\sigma_i$ um 20\% ihres Werte erhöht bzw. gesenkt und damit dann der Datensatz gefittet. Die größte Abweichung vom Referenzwert des regulären Eigenzeitfits wird als systematischer Fehler angenommen. Die Ergenisse finden sich in Tabelle \ref{tab:syst_resolution}.

\begin{table}[hptb]
\centering
\caption{Ergebnisse des Eigenzeitfits bei Variaton der Auflösungsparameter $\sigma_i$ um $\pm 20\%$.}
\label{tab:syst_resolution}
\begin{tabular}{l c c c r@{$\pm$}l r }
\hline \hline
Variation & $sigma_1$ & $sigma_2$ & $sigma_3$ & \multicolumn{2}{c}{$\SJPsi$} & $\Delta\SJPsi$ \\ \hline
$+20\%$ & 0,576 & 0,05275 & 0,1118 & 0,5351 & 0,0626 & 0,0004 \\
$-20\%$ & 0,384 & 0,03517 & 0,0746 & 0,5345 & 0,0625 & -0,0002 \\ \hline \hline
\end{tabular}
\end{table}

Es zeigt sich, dass eine exakte Bestimmung der Auflösung nicht von Nöten ist, da sie im Vergleich zu anderen Systematiken vor allem gegenüber der Flavour-Tagging Kalibration vernachlässigt werden kann. Dennoch wird ein sytematischer Fehler durch die Auflösung mit
\begin{align}
\delta\SJPsi^{\text{Res.}} = 0,0004
\end{align}
assoziiert.

\section{Gesamtsystematik}
Tabelle \ref{tab:syst_gesamt} fasst nochmals alle systematischen Unsicherheiten zusammen. Der Gesamtfehler wird durch quadratische Addition berechnet.

\begin{table}[hptb]
\centering
\caption{Zusammenfassung der systematischen Unsicherheiten}
\label{tab:syst_gesamt}
\begin{tabular}{l c }
\hline \hline
Effekt & $\delta\SJPsi$ \\ \hline
Fitmethode & 0,0033 \\
Flavour-Tagging Kalibration & 0,0331 \\
Eigenzeitakzeptanz & 0,0008 \\
Korrelation Masse $\leftrightarrow$ Eigenzeit & 0,0018 \\ 
Eigenzeitauflösung & 0,0004 \\ \hline 
Gesamt & 0,0333 \\ \hline \hline
\end{tabular}
\end{table}
Es ist deutlich zu erkennen, dass die Kalibration der Flavour-Tagging-Algorithmen die dominierende Systematik ist. Obwohl hier zur Abschätzung der Systematik Werte der 2011-Kalibration genommen werden mussten, wird sich an dieser Tatsache nicht viel ändern, sobald diese Untersuchung mit Werten aus 2012 wiederholt wurde. Der systematischen Fehler von $\delta\SJPsi^{\text{stat.}}=0,0333$ ist nur etwa halb so groß (52,9\%) wie der statistische $\delta\SJPsi^{\text{syst.}}=0,0626$. Damit ist definitiv noch Potential da, die Präzision durch mehr Datennahme zu verbessern.

\chapter{Zusammenfassung}

\begin{thebibliography}{---}
\bibitem{lhcb-paper} LHCb-Analyse, ...
\bibitem{kleinknecht}  K. Kleinknecht, Uncovering ...
\bibitem{pdg-tau} PDG-Wert für Tau \\ \url{http://pdglive.lbl.gov/popupblockdata.brl?nodein=S042T&inscript=Y&fsizein=1&clumpin0=} (Stand: 02.07.2013)
\end{thebibliography}

\printglossaries 
% Selbstständigkeitserklärung
\chapter*{Erklärung}

Ich versichere, dass ich diese Arbeit selbstständig verfasst und keine anderen als die angegebenen Quellen und Hilfsmittel benutzt habe. \\
\vspace{0.2cm}
\begin{flushleft}
Heidelberg, den 19.08.2013,\\
\vspace{2cm}
%Unterschrift
Patrick Fahner
\end{flushleft}


\end{document}


