\documentclass[a4paper]{scrbook}
\usepackage[utf8]{inputenc}
\usepackage[ngerman]{babel}

\usepackage{array}            % Tabellen im Mathemodus (Matrizen)
\usepackage{graphicx}         % Einbindung von Grafiken
\usepackage{braket}           % ermöglicht die Verwendung von Diracs Bra-Ket-Notation
\usepackage{esdiff}           % für Ableitungen


\usepackage{hyperref}         % für automatische Verlinkungen in der pdf
\usepackage{datatool}	      % muss verwendet werden, da glossaries sonst einen Fehler verursacht
\usepackage[toc]{glossaries}  % Glossar

\makeglossaries
\makeindex

% Glossar
\newglossaryentry{OST}{name={OST}, description={Opposite Side Tagger}}
\newglossaryentry{Toy MC}{name={Toy MC}, description={Zur Validierung des Fitters werden zufällig Daten gemäß einer gewünschten Verteilung generiert und im Anschluss gefittet}}


% Definiere Kürzel
\newcommand{\SJPsi}{S_{J/\Psi K_s^0}}
\newcommand{\CP}{$\mathcal{CP}$}
\newcommand{\im}{\mathrm{i}}
\newcommand{\e}{\mathrm{e}}


\begin{document}

% Binde Titelseite ein
\begin{titlepage}
\thispagestyle{empty}
\begin{center}
 
\Large\textbf{Fakultät für Physik und Astronomie\\
Ruprecht-Karls-Universität Heidelberg}

%\vspace{15cm}
\vfill
\normalsize
Bachelorarbeit in Physik\\
eingereicht von\\
\vspace{0.5cm}
\Large\textbf{Patrick Fahner}\\
\normalsize
\vspace{0.5cm}
geboren in Mannheim (Deutschland)\\
\vspace{0.5cm}
\Large\textbf{August 2013}

\newpage
\thispagestyle{empty}
\cleardoublepage
\thispagestyle{empty}
\normalsize
\boldmath
\Huge{\textbf{Messung von $\sin(2\beta)$ im Zerfall \Decaychannel\ am LHCb-Experiment}}
\unboldmath
\vfill
\normalsize
Diese Bachelorarbeit wurde von Patrick Fahner am\\
Physikalischen Institut der Universität Heidelberg\\
unter der Aufsicht von\\
Prof. Dr. Stephanie Hansmann-Menzemer \\
durchgeführt.
\end{center}
\end{titlepage}
\newpage
\hrule
\section*{\abstractname}
Ziel dieser Arbeit ist die Bestimmung des CKM-Winkels $\sin(2\beta)$. Hierzu wird der Zerfallskanal \Decaychannel\ ausgewertet in Daten, die 2012 am LHCb-Detektor bei einer Schwerpunktsenergie von $\sqrt{s}=8\tera\electronvolt$ aufgenommen wurden und einer integrierten Luminosität von ungefähr $2\femto\barn^{-1}$ entsprechen. Als Basis dient eine LHCb-Analyse der im Jahre 2011 aufgenommenen Daten \cite{lhcb-paper}. Das Ergebnis 
\begin{align*}
\sin(2\beta) = 0,711 \pm 0,059(\text{stat.}) \pm 0,033(\text{syst.})
\end{align*}
ist sowohl mit dem Resultat der 2011-Analyse $\sin(2\beta) = 0,72 \pm 0,07 (\text{stat.}) \pm 0,04 (\text{syst.})$ \cite{lhcb-paper} als auch mit dem aktuellen Welt-Mittelwert von $\sin(2\beta) = 0,682 \pm 0,019$ \cite{pdg-average} kompatibel. Ein wichtiger Punkt dieser Arbeit ist die Abschätzung systematischer Effekte. Sie zeigt, dass die systematische Unsicherheit im Wesentlichen von der Kalibration der Algorithmen zur Bestimmung des Produktionsflavours bestimmt ist. \\
\hrule
\vfill
\hrule
\selectlanguage{english}
\hrule
\ZifferPunktAus
\section*{\abstractname}
This thesis provides a measurement of the CKM-angle $\sin(2\beta)$. It evaluates the decay channel \Decaychannel\ of data taken by the LHCb experiment at a center of mass energy of $\sqrt{s}=8\tera\electronvolt$ with an integrated luminosity of about $2\femto\barn^{-1}$. The same analysis strategy as an LHCb analysis of the data taken in 2011 \cite{lhcb-paper} was applied and results
\begin{align*}
\sin(2\beta) = 0.711 \pm 0.059(\text{stat.}) \pm 0.033(\text{syst.})
\end{align*}
which is in good agreement with the 2011 LHCb result $\sin(2\beta) = 0.72 \pm 0.07 (\text{stat.})$ $\pm 0.04 (\text{syst.})$ \cite{lhcb-paper} as well as the world average $\sin(2\beta) = 0.682 \pm 0.019$ \cite{pdg-average}. Furthermore this thesis focuses on the study of systematic effects. The main contribution to the systematic uncertainty is the calibration of the flavour tagging algorithms, which determine the B production flavour. \\ \hrule
\ZifferPunktAn

\selectlanguage{ngerman}




\tableofcontents
\chapter{Das LHCb-Experiment}

\chapter{CP-Verletzung in B-Meson-Systemen}
\section{Diskrete Symmetrietransformationen}
Symmetrien sind in der Physik von zentraler Bedeutung. Gemäß dem Noether-Theorem existiert in der klassischen Physik zu jeder kontinuierlichen Symmetrie eine Erhaltungsgröße. In quantenmechanischen Systemen können wir drei diskrete Symmetrietransformationen betrachten:
\begin{enumerate}
\item \textbf{Parität $\mathcal{P}$:} \\
      Bei der Paritätsoperation wird das Vorzeichen der kartesischen Ortskoordinaten umgekehrt. Dies entspricht einer Punktspigelung.
\item \textbf{Ladungskonjugation $\mathcal{C}$:} \\
      Jedes Teilchen wird durch sein Antiteilchen ersetzt.
\item \textbf{Zeitumkehr $\mathcal{T}$:} \\
      Das Vorzeichen auf der Zeitachse wird umgekehrt. Da in der vorligenden Arbeit allerdings nur die CP-Verletzung gemessen werden soll, wird die Zeitumkehr im folgenden vernachlässigt.
\end{enumerate}
Entgegen der klassischen Intuition konnte Wu 1956 nachweisen, dass die Parität im $\beta$-Zerfall und damit in der schwachen Wechselwirkung nicht erhalten ist. Weitere Experimente zeigen, dass die schwache Wechselwirkung die Parität maximal verletzt: Neutrinos, die nur schwach wechselwirken können, sind stets \glqq linkshändig\grqq (Spin und Impuls antiparallel), Antineutrinos dagegen immer \glqq rechtshändig\grqq (Spin und Impuls parallel). Da der Spin im Gegensatz zum Impuls invariant unter $\mathcal{P}$-Transformation ist, würde diese Operation aus einem linkshändigen Neutrino ein rechtshändiges machen, was in der Nautr nicht realisiert ist.

Damit ist offensichtlich, dass die schwache Wechselwirkung auch die Ladungskonjugation verletzt: Wendet man die $\mathcal{C}$-Transformation auf ein linkshändiges Neutrino an, so erhält man ein linkshändiges Antineutrino. Dieses existiert aber wie bereits erwähnt nicht. Analog gilt die Überlegung auch für Antineutrinos.

\subsection{Scheinbare $\mathcal{CP}$-Invarianz}
Wendet man nun aber die Transformationen $\mathcal{P}$ und $\mathcal{C}$ direkt hintereinander an, so ergibt sich zunächst kein Widerspruch zur Natur (siehe Abb. \ref{fig:cp_invarianz}). Aus einen linkshändigen Neutrino wird ein rechtshändiges Antineutrino. Im Jahre 1964 wurde dann allerdings im Zerfall neutraler K-Mesonen erstmals $\mathcal{CP}$-Verletzung nachgewiesen. \cite{kleinknecht}

\begin{figure}[hptb]
\centering
\includegraphics[width = 0.8\textwidth]{cp_invarianz}
\caption{Scheinbare $\mathcal{CP}$-Invarianz: Während eine reine $\mathcal{P}$- oder $\mathcal{C}$-Transformation zu in der Natur nicht realisierten Zuständen führt, scheint es bei der kombinierten $\mathcal{CP}$-Transformation keinen Widerspruch zu geben (dünne Pfeile: Impulsausrichtung, dicke Pfeile: Spinausrichtung).}
\label{fig:cp_invarianz}
\end{figure}



\section{\CP-Verletzung in der Mischung}
Die Flavoureigenzustände $\Ket{B^0} = \Ket{\overline{b}d}$ und $\Ket{\overline{B^0}} = \Ket{b\overline{d}}$ entsprechen nicht den Masseneigenzuständen. Wir definieren daher die normierten Zustände
\begin{align}
\Ket{B_h} = p \Ket{B^0} - q \Ket{\overline{B^0}} \label{eq:b_heavy}\\ 
\Ket{B_l} = p \Ket{B^0} + q \Ket{\overline{B^0}} \label{eq:b_light}\\
\text{mit} \quad |p|^2 + |q|^2 = 1
\end{align}
welche eine definierte Masse und Zerfallsbreite besitzen. Sie sind auch Eigenzustände eines nicht-hermiteschen Hamiltonoperators (Nichthermitizität wegen des möglichen Zerfalls der Teilchens). Dieser setzt sich zusammen aus den hermiteschen Massenoperatoren $M$ und $\Gamma$. Notieren wir die lineare Superposition der Zustände \ref{eq:b_heavy} und \ref{eq:b_light} als $\begin{pmatrix} p \\ q \end{pmatrix}$, so nimmt die zeitabhängige Schrödingergleichung die Form
\begin{align}
\im \diff{}{t}\begin{pmatrix} p \\ q \end{pmatrix} = \left(M - \frac{\im}{2} \Gamma\right) \begin{pmatrix} p \\ q \end{pmatrix}
\end{align}
an und führt zur folgenden zeitlichen Entwicklung der Zustände:
\begin{align}
\nonumber \Ket{B_{h/l}(t)} &= \e^{-\im m_{h/l}t-\frac{1}{2}\Gamma_{h/l}t}\Ket{B_{h/l}(0)} \\
                           &= \e^{-\gamma_{h/l}t}(p\Ket{B^0} \mp q\Ket{\overline{B^0}}) \\
&\text{mit} \quad \gamma_{h/l} = \im m_{h/l}+\frac{\Gamma_{h/l}}{2}
\end{align}
Hierbei ist $\gamma_{h/l}$ so definiert, dass $-\im\gamma_{h/l} = m_{h/l}-\frac{\im}{2}\Gamma_{h/l}$ die Eigenwerte des Hamiltonoperators $\mathcal{H} := \left(M - \frac{\im}{2} \Gamma\right)$ sind. Umgeschrieben auf die Flavoureigenzustände erhält man:
\begin{align}
\nonumber \Ket{B^0(t)} &= \frac{1}{2p}\left(\Ket{B_h} + \Ket{B_l}\right) \\
                       &= \frac{1}{2}\left[ (\e^{-\gamma_h t}+\e^{-\gamma_l t})\Ket{B^0} - \frac{q}{p}(\e^{-\gamma_h t}-\e^{-\gamma_l t})\Ket{\overline{B^0}}\right] \label{eg:b(t)}
\end{align}
Die Wahrscheinlichkeit für den Übergang eines $\Ket{B^0}$ (zum Zeitpunkt $t=0$) in ein $\Ket{\overline{B^0}}$ beträgt:
\begin{align}
\nonumber P(B^0\rightarrow\overline{B^0})(t) &= |\Braket{\overline{B^0}|B^0(t)}|^2 \\
                                        &= \frac{1}{4} \left|\frac{q}{p}\right|^2 \left[\e^{-\Gamma_h t} + \e^{-\Gamma_l t} - 2\e^{-\frac{1}{2}(\Gamma_h + \Gamma_l) t}\cos(\Delta m_d t)\right] \\
&\text{mit} \quad \Delta m_d = m_h - m_l
\end{align}

Analog gilt für die Übergangswahrscheinlichkeit eines $\Ket{\overline{B^0}}$ in ein $\Ket{B^0}$:
\begin{align}
P(\overline{B^0}\rightarrow B^0)(t) = \frac{1}{4} \left|\frac{p}{q}\right|^2 \left[\e^{-\Gamma_h t} + \e^{-\Gamma_l t} - 2\e^{-\frac{1}{2}(\Gamma_h + \Gamma_l) t}\cos(\Delta m_d t)\right] 
\end{align}

Es kommt daher in der Mischung zur \CP-Verletzung, wenn die Oszillation ungleichmäßig verläuft, anders ausgedrückt:
\begin{align}
\text{\CP-Verletzung in der Mischung} \qquad \Longleftrightarrow \qquad \left|\frac{p}{q}\right| \neq 1 
\end{align}

\section{Direkte \CP-Verletzung}
Die Zerfallsamplituden der neutralen $B^0$-Mesonen in einen Endzustand $\Ket{f}$ bzw. seinen \CP-konjugierten Zustand $\Ket{\overline{f}}$ sind definiert als
\begin{alignat}{2}
\nonumber A_f &= \Braket{f|\mathcal{H}|B^0}, && \qquad A_{\overline{f}} = \Braket{\overline{f}|\mathcal{H}|B^0}, \\
          \overline{A_f} &= \Braket{f|\mathcal{H}|\overline{B^0}}, && \qquad  \overline{A_{\overline{f}}} = \Braket{\overline{f}|\mathcal{H}|\overline{B^0}}. \label{eq:decay_amplitudes}
\end{alignat}
Dabei bezeichnet $\mathcal{H}$ einen Hamiltonoperator der schwachen Wechselwirkung. Ist \CP erhalten, dann sollten die Zerfallsraten, ergo auch die Zerfallsamplituden eines $B^0$ nach $f$ sowie eines $\overline{B^0}$ nach $\overline{f}$ gleich sein. Dies bedeutet:
\begin{align}
\text{Direkte \CP-Verletzung} \qquad \Longleftrightarrow \qquad \frac{|A_f|}{|\overline{A_{\overline{f}}}|} \neq 1 \quad \text{bzw.} \quad \frac{|\overline{A_f}|}{|A_{\overline{f}}|} \neq 1
\end{align}


\section{\CP-Verletzung in der Interferenz}
Die Zustände \ref{eq:b_heavy} und \ref{eq:b_light} haben eine nahezu gleiche Anzahl an Zerfällskanäle. Dies hat zur Folge, dass die Lebensdauern des schweren und leichten Zustands innerhalb weniger Prozent gleich sind:
\begin{align}
\Gamma := \Gamma_h = \Gamma_l \label{eq:Gamma}
\end{align}

Weiterhin sagt das Standard Modell nur eine kleine \CP-Verletzung in der \Bd - \Bdbar - Mischung voraus, sodass
\begin{align}
\left|\frac{p}{q}\right| = 1 \qquad \text{in} \mathcal{O}(10^{-3}). \label{eg:pq_approx}
\end{align}

Für das B-Meson-System bleibt daher nur die Möglichkeit der \CP-Verletzung in der Interferenz von Mischung und direktem Zerfall. Der in dieser Arbeit betrachtete Zerfallskanal $B_d^0 \rightarrow J/\Psi K_s^0$ hat einen \CP-Eigenzustand als Endzustand (\CP $\Ket{\JPsi\Kshort} = -\Ket{\JPsi\Kshort}$). In Anlehnung an \ref{eq:decay_amplitudes} sind die Zerfallsamplituden hier definiert als
\begin{align}
\nonumber A_f := \Braket{f|B^0(t)}, \qquad \overline{A_{f}} := \Braket{f|\mathcal{H}|\overline{B^0}}
\end{align}

Mit Blick auf die Zerfallsamplituden der Masseneigenzustände wird die komplexe Größe
\begin{align}
\lambda := \frac{q\overline{A_f}}{pA_f} \label{eq:lambda}
\end{align}
definiert. Ausgehend von Gleichung \ref{eg:b(t)} sowie mit Hilfe fer Gleichungen (\ref{eq:Gamma}), (\ref{eg:pq_approx}) und (\ref{eq:lambda}) gilt für die Zerfallsrate eines anfänglich reinen \Bd-Zustands:
\begin{align}
\nonumber \Gamma (B^0 \rightarrow \JPsi\Kshort) &= \frac{1}{4}\left| (\e^{-\gamma_h t}+\e^{-\gamma_l t})A_f - \frac{q}{p}(\e^{-\gamma_h t}-\e^{-\gamma_l t})\overline{A_f}\right|^2 \\
&= \frac{1}{2} \left|A_f\right|^2\e^{-\Gamma t} \left[1+|\lambda|^2 + (1-|\lambda|^2)\cos(\Delta m_d t) - 2\mathrm{Im}(\lambda)\sin(\Delta m_d t)\right]
\end{align}
Analog:
\begin{align}
\Gamma (\overline{B^0} \rightarrow \JPsi\Kshort) &= \frac{1}{2} \left|A_f\right|^2\e^{-\Gamma t} \left[1+|\lambda|^2 -(1-|\lambda|^2)\cos(\Delta m_d t) + 2\mathrm{Im}(\lambda)\sin(\Delta m_d t)\right]
\end{align}

Damit kann die vom Standard Modell prognostizierte \CP-verletzende Asymmetrie 
\begin{align}
\mathcal{A}_{\text{\CP}} &= \frac{\Gamma (\overline{B^0} \rightarrow \JPsi\Kshort) - \Gamma (B^0 \rightarrow \JPsi\Kshort)}{\Gamma (\overline{B^0} \rightarrow \JPsi\Kshort) + \Gamma (B^0 \rightarrow \JPsi\Kshort)} \\
&= -\frac{1-|\lambda|^2}{1+|\lambda|^2}\cos(\Delta m_d t) + \frac{2\mathrm{Im}(\lambda)}{1+|\lambda|^2}\sin(\Delta m_d t) \\
&=: \CJPsi \cos(\Delta m_d t) + \SJPsi \sin(\Delta m_d t)
\end{align}
berechnet werden und vereinfacht sich - da $\Ket{\JPsi\Kshort}$ ein \CP-Eigenzustand ist, gilt $|\lambda| = 1$ - hier zu
\begin{align}
\mathcal{A}_{\text{\CP}} = \mathrm{Im}(\lambda)\sin(\Delta m_d t) .
\end{align}

Damit kann im B-Meson-System, insbesondere im Zerfall $B_d^0 \rightarrow J/\Psi K_s^0$ durch Messung der Asymmetrie-Amplitude $\SJPsi$ \CP-Verletzung in der Interferenz gemessen werden.

\begin{align}
\text{\CP-Verletzung in der Interferenz} \qquad \Longleftrightarrow \qquad \SJPsi = \mathrm{Im}(\lambda)\neq 0
\end{align}

\section{CKM-Formalismus}

\chapter{Datenselektion}

\chapter{Analyse / Fit}
\section{Fitmethode SFit}
\section{Wahrscheinlichkeitsdichtefunktion}
\section{Fitergebnis} \label{kap:fitergebnis}
Wir erhalten schließlich:
\begin{align}
\SJPsi = xxx \pm xxx     \label{eq:fit_result}
\end{align}

\chapter{Abschätzung systematischer Fehler}
Der Fitter liefert uns zwar eine statistische Unsicherheit auf $\SJPsi$, allerdings ist eine Betrachtung der Systematik unerlässlich. Im Folgenden wird daher der Einfluss einiger Effekte auf das Fitergebnis untersucht und anschließend der systematische Fehler abgeschätzt.

\section{Fit Bias}
Die hier verwendete Maximum-Likelihood-Methode hat zwar den schönen Vorteil, dass das Fitergebnis nicht vom Binning abhängt, es ist jedoch nicht von vornherein ausgeschlossen, dass sie das Ergebnis verfälscht (einen sog. Bias produziert). Daher wird eine Toy Monte Carlo - Studie (kurz: Toy MC) durchgeführt. Dabei werden Daten zufällig nach einer Verteilung mit den gewünschten Parametern generiert und im Anschluss gefittet. Für eine gute Abschätzung des Bias dienen die Ergebnisse des Daten-Fits aus Tabelle xxxxxxx als Anhaltspunkt. Es werden pro Toy 20000 Teilchen generiert mit einem Signalanteil von 42,3\%.

\begin{figure}[hptb]
\centering
\includegraphics[width = \textwidth]{ampl_both}
\caption{Verteilung der aus der Toy MC Studie erhaltenen Amplituden $\SJPsi$ (links) sowie die dazugehörigen Pulls (rechts)}
\label{fig:fit_bias}
\end{figure}

Abbildung \ref{fig:fit_bias} zeigt sowohl die Verteilung der gefitteten Amplitude und die Pulls, die sich mittels $\frac{\SJPsi^{gefittet} - \SJPsi^{generiert}}{\sigma^{gefittet}}$ berechnen. Es lassen sich hierbei zwei Dinge beobachten:
\begin{enumerate}
    \item An der Verschiebung des Pull-Mittelwertes von der Null sieht man deutlich, dass es einen kleinen, aber signifikanten Bias gibt. Indem wir diesen Bias mit der statistischen Unsicherheit aus unserem Fitergebnis (siehe Gl. \ref{eq:fit_result}) multiplizieren erhalten wir eine Abschätzung der aus der Fitmethode resultierenden Unsicherheit:
        \begin{align}
        \delta\SJPsi^{Fit} = 0,059 \cdot 0,07 = 0,00413
        \end{align}

    \item Mit einem $\sigma = 0,957 \pm 0,005$ ist die Pull-Verteilung signifikant zu schmal. Dies bedeutet, dass der Fit den statistischen Fehler überschätzt. Das Problem tritt auf, sobald man in den Toys Untergrund miteinbezieht. Es ist bekannt, dass die verwendete SFit-Methode die Fehlerpropagation (gerade bei Untergrund) nicht korrekt ausführt. Es wurde daher eine Fehlerkorrektur implementiert, aber auch dieser handelt es sich nur um eine Näherung.
\end{enumerate}

\paragraph{Ursachen des Bias}
Weitere Toy MC Studien zeigen, dass die Behandlung des Untergrundes zu einem Bias führt. Generiert man nämlich nur Signal, ist der Mittelwert kompatibel zur Null (siehe Abb. \ref{fig:toys_no_bkg}).

\begin{figure}[hptb]
\centering
%\includegraphics[width=\textwidth]{}
\caption{Toy MC Studie mit reinem Signal ohne Untergrund. Es kommt zu keinem signifikanten Bias. Links: Verteilung der erhaltenen Amplitude, Rechts: Pull-Verteilung.}
\label{fig:toys_no_bkg}
\end{figure}

Die Vermutung ist, dass zu wenig Statistik im Fit die eigentliche Ursache für den Bias ist. Daher wurden weitere Toy MC Studien mit unterschiedlicher Anzahl an Teilchen pro Toy durchgeführt. Die Ergebnisse sind in Tabelle \ref{tab:fit_bias_events} aufgeführt und in Abbildung \ref{fig:fit_bias_events} nochmals visualisiert.

\begin{table}[hptb]
\centering
\caption{Toy MC Studien mit unterschiedlicher Anzahl an generierten Events pro Toy. Genannt wird der Mittelwert $\mu$ der $\SJPsi$-Pull-Verteilung}
\label{tab:fit_bias_events}
\begin{tabular}{cr@{$\pm$}l}
\hline \hline 
Teilchen pro Toy & \multicolumn{2}{c}{$\mu$}  \\ \hline
20000            &  xxx & xxx \\
50000            &  xxx & xxx \\
100000           &  xxx & xxx \\
200000           &  xxx & xxx \\ 
\hline \hline
\end{tabular}
\end{table}

\begin{figure}[hptb]
\centering
%\includegraphics[width=\textwidth]{}
\caption{Visualisierung der Werte aus Tab. \ref{tab:fit_bias_events}}
\label{fig:fit_bias_events}
\end{figure}

\section{Tagging Kalibration}
Im Fit wird bei den Parametern der Tagging Kalibration nur der statistischen Fehler berücksichtigt. Es soll nun an dieser Stelle der Einfluss der statistischen Unsicherheiten abgeschätzt werden. \\
Die Korrekturparameter $p_0$ und $p_1$ für die Fehlerwahrscheinlichkeit des \gls{OST} sind gegeben durch
\begin{align}
p_0 &= 0,392 \pm 0,0017\ \text{(stat.)} \pm 0,0076\ \text{(syst.)} \\
p_1 &= 1,035 \pm 0,021\ \text{(stat.)} \pm 0,0076\ \text{(syst.)}.
\end{align}

\paragraph{Variation der Parameter in den Daten}
Zunächst werden die Startwerte der Parameter $p_0$ und $p_1$ variiert, indem man jeweils den systematischen Fehler der Parameter addiert bzw. subtrahiert und dann den Fit auf die Daten durchührt. Für alle vier Kombinationen wird dann die Abweichung vom regulären Fitergebnis für $\SJPsi$ berechnet. Der Referenzwert aus dem Fit beträgt
\begin{align}
\SJPsi = 0,625 \pm 0,069
\end{align}

\begin{table}[hptb]
\centering
\caption{Variation des Fitergebnisses für $\SJPsi$ bei Veränderung der Startwerte für $p_0$ und $p_1$ $\pm$ ihrer statistischen Unsicherheiten}
\label{tab:syst_fit_calib_data}
$\begin{array}{cc|c|c}
\hline\hline
p_0            & p_1           & \SJPsi          & \Delta\SJPsi   \\ \hline
0,392 - 0,0076 & 1,035 - 0,012 & 0,599 \pm 0,067 & -0,026 \pm xxx \\
0,392 + 0,0076 & 1,035 - 0,012 & 0,661 \pm 0,072 & 0,036 \pm xxx \\
0,392 - 0,0076 & 1,035 + 0,012 & 0,592 \pm 0,066 & -0,033 \pm xxx \\
0,392 + 0,0076 & 1,035 + 0,012 & 0,651 \pm 0,071 & 0,026 \pm xxx \\
\hline\hline
\end{array}$
\end{table}

Die Ergebnisse sind Tabelle \ref{tab:syst_fit_calib_data} zu entnehmen. Die größte Abweichung beträgt hier $\Delta\SJPsi = 0,036$.

\paragraph{Variation der Parameter in \gls{Toy MC}}
Eine weitere Möglichkeit der Abschätzung besteht darin, sich entsprechende Toys zu generieren und diese dann zu fitten. Im Folgenden werden bei der Toy Generierung die Parameter $p_0$ und $p_1$ um ihre systematische Unsicherheiten variiert, der Fit dann allerdings mit den ursprünglichen Parameterwerten durchgeführt. Als Referenzwert generieren und fitten wir toys mit den ursprünglichen Parameterwerten $p_0$ und $p_1$ sowie $\SJPsi = 0.75$ und erhalten hierfür:

\begin{align}
\SJPsi = 0,75527 \pm xxx
\end{align}

\begin{table}[hptb]
\centering
\caption{Variation des Fitergebnisses für $\SJPsi$ bei Veränderung der Parameterwerte $p_0$ und $p_1$ $\pm$ ihrer statistischen Unsicherheiten bei der Generierung von Toys}
\label{tab:syst_fit_calib_toys}
$\begin{array}{cc|c|c}
\hline\hline
p_0            & p_1           & \SJPsi          & \Delta\SJPsi   \\ \hline
0,392 - 0,0076 & 1,035 - 0,012 & 0,782 \pm xxx & 0,027 \pm xxx \\
0,392 + 0,0076 & 1,035 - 0,012 & 0,719 \pm xxx & -0,036 \pm xxx \\
0,392 - 0,0076 & 1,035 + 0,012 & 0,788 \pm xxx & 0,032 \pm xxx \\
0,392 + 0,0076 & 1,035 + 0,012 & 0,727 \pm xxx & -0,028 \pm xxx \\
\hline\hline
\end{array}$
\end{table}

Die Ergebnisse sind Tabelle \ref{tab:syst_fit_calib_toys} zu entnehmen. Die größte Abweichung beträgt hier betragsmäßig ebenfalls $\Delta\SJPsi = 0,036$. Daher schätzen wir den systematischen Fehler durch die Tagging Kalibrierung auf 

\begin{align}
s_{tag_calib} = 0.036 \ .
\end{align}

\section{Einfluss der zeitlichen Akzeptanz}

\section{Gleichförmigkeit der Massenverteilung}

\chapter{Zusammenfassung}

\begin{thebibliography}{---}
\bibitem{kleinknecht}  K. Kleinknecht, Uncovering ...
\end{thebibliography}

\printglossaries 
% Selbstständigkeitserklärung
\chapter*{Erklärung}

Ich versichere, dass ich diese Arbeit selbstständig verfasst und keine anderen als die angegebenen Quellen und Hilfsmittel benutzt habe. \\
\vspace{0.2cm}
\begin{flushleft}
Heidelberg, den 19.08.2013,\\
\vspace{2cm}
%Unterschrift
Patrick Fahner
\end{flushleft}


\end{document}


