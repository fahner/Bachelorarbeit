\chapter{Analyse} \label{kap:analyse}
Um aus einem Datensatz den \glqq wahren\grqq\ Wert diverser Parameter abzuschätzen, gibt es verschiedene Möglichkeiten. In dieser Analyse wird die Methode sFit verwendet. Diese stellt eine modifizierte Variante des \glqq Unbinned Maximum Likelihood\grqq\ Fits dar. Unbinned meint, dass das Fitergebnis nicht von der Wahl der Säulen (engl. bins) eines Histogramms abhängt. Die Modifikation des Fits besteht in der Verwendung der aus der \SPlot-Technik bekannten Gewichten, den sWeights. Dadurch ist es nicht nötig, den Untergrund zu modellieren, da dieser aus statistischen Gründen annihiliert wird.

\section{Maximum Likelihood Funktion}
Die Maximum Likelihood Methode ist eine weit verbreitete Methode, um Parameter abzuschätzen. Für eine gegebene Wahrscheinlichkeitsdichtefunktion (WDF) $\mathcal{P}(\vec{x}_e;\vec{\lambda})$ mit einem unbekannten Satz Parametern $\vec{\lambda}$ und $N$ unabhängigen Messungen $\vec{x}_e$ ist die Likelihood-Funktion als
\begin{align}
\mathcal{L}(\vec{\lambda}) = \prod_{e=1}^N \mathcal{P}(\vec{x}_e;\vec{\lambda})
\end{align}
definiert. Der Satz an Parametern, der $\mathcal{L}$ maximiert, gilt als beste Abschätzung von $\vec{\lambda}$. In der Praxis jedoch minimiert man äquivalent $-\ln\mathcal{L}$. Gewöhnlicherweise berücksichtigt man möglichen Untergrund, indem man die WDF in einen Signal- und Untergrundanteil aufteilt:
\begin{align}
\mathcal{P}(\vec{x}_e;\vec{\lambda}) = f_{sig}\mathcal{P}_{sig}(\vec{x}_e;\vec{\lambda}) + (1-f_{sig})\mathcal{P}_{bkg}(\vec{x}_e). \label{eq:likelihood_sig_bkg}
\end{align}
$f_{sig}$ bezeichnet hierbei den Signalanteil, $\mathcal{P}_{sig}, \mathcal{P}_{bkg}$ die WDF des Signals bzw. Untergunds. Die Schwierigkeit besteht nun darin, den Untergrund geeignet zu modellieren. Dazu bedarf es MonteCarlo-Studien oder der Verwendung separater Massenseitenbänder. \cite{sfit}

\section{Fitmethode sFit} \label{kap:sfit}
Der sFit bietet eine Möglichkeit, ohne genaue Kenntnis des Hintergrunds die wahre Verteilung des Signalanteils von $\vec{x}$ zu rekonstruieren. Dazu bedarf es einer weiteren Variable $\vec{y}$, die vollkommen unkorreliert ist, also sowohl für Signal als auch Untergrund. In dieser Analyse wird später $\vec{y} = y = M($\Bd$)$ die rekonstruierte Masse der \Bd-Mesonen sein, $\vec{x}^T = (t,d,\eta)^T$, die Variablen, die zur Bestimmung von $\SJPsi$ notwendig sind. Was diese im Einzelnen bedeuten, wird später behandelt.

Sei $N_s$ die Zahl an Signal- und $N_b$ die Zahl an Untergrund-Ereignissen eines Datensatzes. Die Verteilungen von Signal und Untergund der Variablen $y$ seien mit $F_s(y)$ bzw. $F_b(y)$ bezeichnet und all diese vier Größen seien bekannt. Dann stellt die \SPlot-Technik (\cite{splot}) mit den sogenannten \glqq sWeights\grqq 
\begin{align}
W_s(y) = \frac{V_{ss}F_s(y)+V_{sb}F_b(y)}{N_sF_s(y)+N_bF_b(y)}
\end{align} 
einen Formalismus zur Verfügung, um durch Gewichtung der Ereignisse das Signal vom Untergrund zu bereinigen. Die Matrix $V_{ij}$ bezeichnet dabei das Inverse der Kovarianzmatrix
\begin{align}
V_{ij}^{-1} = \sum_{e=1}^N \frac{F_i(y_e)F_j(y_e)}{(N_sF_s(y_e)+N_bF_b(y_e))^2}.
\end{align}
In der \SPlot-Technik werden die Gewichte $W_s(y_e)$ berechnet und anschließend ein Histogramm mit den Messungen $x_e$ mit der entsprechenden Gewichtung $W_s(y_e)$ gefüllt, um die Signalverteilung von x zu erhalten. Beim sFit wird nun die Likelihood Funktion gemäß
\begin{align}
\mathcal{L}_W(\vec{\lambda}) = \prod_{e=1}^N \mathcal{P}(\vec{x}_e;\vec{\lambda})^{W_s(y_e)}
\end{align}
gewichtet. Die Erwartung ist, dass der Untergrundanteil auf statistischer Grundlage eliminiert wird und der wahre Wert von $\vec{\lambda}$ durch Maximierung von $\mathcal{L}_W(\vec{\lambda})$ abgeschätzt werden kann \cite{sfit}.

\section{Fit der Massenverteilung und Bestimmung der sWeigths} \label{kap:massenfit}
Wie bereits in Kapitel \ref{kap:sfit} erwähnt, wird die rekonstruierte Masse zur Berechnung der sWeights herangezogen. Dazu wird ein klassischer Maximum Likelihood Fit durchgeführt, d.h. Signal und Untergrund werden gemäß Gleichung \ref{eq:likelihood_sig_bkg} gesondert beschrieben. Für den Signalteil der Massenverteilung wird ein doppelter Gauß der Form
\begin{align}
\mathcal{P}_{m;S}(m;\vec{\lambda}_{m;S}) = f_{S,m}\mathcal{G}(m;m_{\text{\Bd}},\sigma_{m,1}) + (1-f_{S,m})\mathcal{G}(m;m_{\text{\Bd}},\sigma_{m,2})
\end{align}
mit gemeinsamen Mittelwert $m_{\text{\Bd}}$, unterschiedlichen Breiten $\sigma_{m,1}$ und $\sigma_{m,2}$ sowie dem relativen Beitrag $f_{S,m}$ der beiden Gauß-Kurven angenommen. Die Normierung ist dabei bereits in $\mathcal{G}$ enthalten. Der Untergrund wird durch die Exponentialfunktion
\begin{align}
\mathcal{P}_{m;B}(m;\vec{\lambda}_{m;B}) = \frac{1}{\mathcal{N}_{m;B}}\e^{-\alpha_m m}
\end{align}
modelliert. $\mathcal{N}_{m;B}$ bezeichnet dabei die Normierung auf den im Fit verwendeten Massenbereich $m \in [5230,5330]\mega\electronvolt$. In \cite{lhcb-paper} wurde anhand von MonteCarlo-Studien gezeigt, dass in diesem Fitbereich kein störender Untergund vorhanden ist, der besonders berücksichtigt werden müsste. Damit lautet die gesamte Wahrscheinlichkeitsdichtefunktion der Massenverteilung
\begin{align}
\mathcal{P}_{m}(m;\vec{\lambda}_{m}) = f_{sig}\mathcal{P}_{m;S}(m;\vec{\lambda}_{m;S}) + (1-f_{sig})\mathcal{P}_{m;B}(m;\vec{\lambda}_{m;B}) \label{eq:pdf_masse},
\end{align}
wobei $f_{sig}$ den Anteil des Signals angibt. Der Fit liefert für den Parametersatz $\vec{\lambda}_{m}^T = (f_{sig}, f_{S,m}, m_{\text{\Bd}}, \sigma_{m,1},\sigma_{m,2}, \alpha_m)^T$ die in Tabelle \ref{tab:fit_masse} aufgeführten Resultate. Alle Parameter wurden dabei im Fit laufen gelassen. 
\begin{table}[hptb]
\centering
\caption{Ergebnisse des Massenfits zur Bestimmung der sWeights}
\label{tab:fit_masse}
$\begin{array}{llr@{\pm}l}
\hline \hline
\multicolumn{2}{l}{\text{Parameter}} & \multicolumn{2}{c}{\text{Wert}}\\ \hline
f_{sig} &  & 0,628 & 0,017\\
f_{S,m} &  & 0,59 & 0,23 \\
m_{\text{\Bd}} & [\mega\electronvolt]& 5281,55 & 0,12\\
\sigma_{m,1} & [\mega\electronvolt]  & 8,14 & 0,98\\
\sigma_{m,2} &[\mega\electronvolt] & 14,3 & 3,4  \\
\alpha_m &[\mega\electronvolt^{-1}]& 0,00143 & 0,00046  \\ \hline
\end{array}$
\end{table}

Des Weiteren zeigt Abbildung \ref{fig:fit_masse} die Massenverteilung mit Fit, die dazugehörigen Pulls sowie die berechneten sWeights. Pulls sind die auf den Fehler des Messwerts normierten Residuen. Für eine beliebige Messgröße $y(x)$ werden sie berechnet über
\begin{align}
pull(x) = \frac{y_{gemessen}-y_{gefittet}}{\sigma_y}.
\end{align}
Man erwartet, dass die Pulls bei einem \glqq guten\grqq\ Fit zufällig und gaußverteilt um die Nulllinie streuen.

\begin{figure}[hptb]
\centering
\includegraphics[width=0.9\textwidth]{mass_fit}
\caption{Ergebnis des Massenfits. Gezeigt wird die Verteilung der rekonstruierten \Bd-Mesonen inklusive des Fits (oben), die zum Fit gehörigen Pulls (mitte) sowie ein zweidimensionales Histogramm mit den berechneten sWeights (unten).}
\label{fig:fit_masse}
\end{figure}


\section{Fit der Eigenzeitverteilung} \label{kap:eigenzeitverteilung}
In diesem Abschnitt soll nun die Wahrscheinlichkeitsdichtefunktion der \Bd-Ei\-gen\-zeit\-ver\-tei\-lung entwickelt werden, die letztendlich zur Bestimmung der Asymmetrie-Amplitude $\SJPsi$ verwendet wird. Aus den Gleichungen (\ref{eq:bd}) und (\ref{eq:bdbar}) geht für $|\lambda_f|=1$ die theoretische Eigenzeitverteilung für ein \Bd\ bzw. \Bdbar\ hervor:
\begin{align}
\mathcal{P}_{\text{wahr}}(t, d_{\text{wahr}}) = \frac{1}{\mathcal{N}_t}\e^{-t/\tau}\left[1-d_{\text{wahr}}\SJPsi\sin(\sinarg)\right].
\end{align}
Durch die Einführung der Variablen $d_{\text{wahr}}$ wurden beide Verteilungen zu einer zusammengefasst. Dieses steht für den wahren Flavour des Mesons zum Zeitpunkt $t=0$. Ein anfängliches \Bd\ wird dabei durch $d_{\text{wahr}}=1$ beschrieben, ein \Bdbar\ durch $d_{\text{wahr}}=-1$. Die Normierung ist so gewählt, dass die Bedingung
\begin{align}
\sum_{d_{\text{wahr}}}\int_{t_{min}}^{t_{max}}\mathrm{d}t\mathcal{P}_{\text{wahr}}(t, d_{\text{wahr}}) = 1
\end{align}
erfüllt wird. Aufgrund zahlreicher detektor- und experimentbedingter Effekte muss $\mathcal{P}_{\text{wahr}}(t, d_{\text{wahr}})$ modifiziert werden, bevor es im Fit verwendet werden kann.

\subsection{Produktionsasymmetrie}
Im Experiment werden \Bd- und \Bdbar-Mesonen nicht in exakt gleicher Zahl produziert. Über die Produktionsraten $R_{\text{\Bdbar}}$ für ein \Bdbar\ bzw. $R_{\text{\Bd}}$ für ein \Bd\ ist die Produktionsasymmetrie definiert durch:
\begin{align}
\mu = A_P = \frac{R_{\text{\Bdbar}}-R_{\text{\Bd}}}{R_{\text{\Bdbar}}+R_{\text{\Bd}}}.
\end{align}
Anhand dieser Definition muss der Anteil an \Bd\ bzw. \Bdbar\ an der gesamten WDF gewichtet werden. Unter Verwendung des Kronecker-Deltas $\delta_{ij}$ lässt sich die WDF daher schreiben als:
\begin{align}
\nonumber \widetilde{\mathcal{P}}_{\text{wahr}}(t, d_{\text{wahr}}) &= \delta_{d_{\text{wahr}},1}(1-\mu)\mathcal{P}_{\text{wahr}}(t, 1) + \delta_{d_{\text{wahr}},-1}(1+\mu)\mathcal{P}_{\text{wahr}}(t, -1) \\
\nonumber &= (1-d_{\text{wahr}}\mu)\mathcal{P}_{\text{wahr}}(t, d_{\text{wahr}}) \\
&= \frac{1}{\mathcal{N}_t}\e^{-t/\tau}\left[1-d_{\text{wahr}}\mu - (d_{\text{wahr}}-\mu)\SJPsi\sin(\sinarg)\right].
\end{align}
Der Wert der Produktionsasymmetrie $\mu$ wurde in einigen Studien gemessen. Es wird mit 
\begin{align}
\mu = -0,015 \pm 0,013
\end{align}
derselbe Wert wie in der LHCb Analyse aus 2011 \cite{lhcb-paper} verwendet.

\subsection[Bestimmung des Produktionsflavours der \Bd-Mesonen (Flavour Tagging)]{Bestimmung des Produktionsflavours der \boldmath\Bd-Mesonen\unboldmath\ (Flavour Tagging)} \label{kap:tagging}
Die Messung der indirekten \CP-Verletzung setzt voraus, dass der anfängliche Flavour des \Bd-Mesons bekannt ist. Den Prozess, den Produktionsflavour eines \Bd-Mesons zu bestimmen, also zu überprüfen, ob ein ein $b$ oder $\overline{b}$ Quark vorlag, nennt man Flavour Tagging. Hierzu werden sogenannte Tagging Algorithmen angewandt, die allerdings nur mit einer bestimmten Effizienz arbeiten. Von $N$ Kandidaten kann bei $N_U$ Kandidaten kein Anfangsflavour zugeordnet werden, bei $N_W$ ist er falsch und bei $N_R$ ist er richtig. Ein Maß für die Güte des Algorithmus ist die Tagging Effizienz
\begin{align}
\epsilon_{\text{tag}} = \frac{N_R+N_W}{N_R+N_W+N_U}
\end{align}
und die sog. Mistagwahrscheinlichkeit
\begin{align}
\omega = \frac{N_W}{N_R+N_W},
\end{align}
die die Wahrscheinlichkeit angibt, den Signalkandidaten den falschen Flavour zuzuordnen. Die Größe die es bei solchen Algorithmen zu maximieren gilt, ist die effektive Tagging Effizienz
\begin{align}
\epsilon_{\text{eff}} = \epsilon_{\text{tag}}(1-2\omega)^2 =: \epsilon_{\text{tag}} \mathcal{D}^2.
\end{align}
$\mathcal{D}$ wird auch Verunreinigungsfaktor genannt. Die effektive Tagging Effizienz gibt an, auf wie viel Prozent die Statistik effektiv reduziert wird. Beträgt zum Beispiel $\epsilon_{\text{eff}} = 2\%$, dann ist es genau so gut 50000 Signalkandidaten mit fehlerbehaftetem Flavour Tagging oder 1000 Signalkandidaten, bei denen der wahre Produktionsflavour bekannt ist. Bei dem in dieser Arbeit verwendeten Flavour Tagging Algorithmus handelt es sich um einen sog. Opposite Side Tagger (OST). Dieser nutzt aus, dass die meisten $b$ Quarks in Quark-Antiquark Paaren erzeugt werden. Dabei rekonstruiert der OST pertiell die Zerfallsreste des entsprechenden Quark-Partners des \Bd-Mesons. Der Algorithmus berechnet aus kinematischen und geometrischen Daten mit Hilfe eines neuronalen Netzes eine Fehlerwahrscheinlichkeit $\eta^{OS} \in [0;0,5]$ für seine Flavour-Zuweisung, die im folgenden mit $d$ bezeichnet wird. \cite{lhcb-paper}

Der Output des neuronalen Netzes $\eta^{OS}$ muss allerdings noch auf anderen Zerfallskanälen kalibriert werden. Dies ist jedoch nicht Bestandteil dieser Arbeit, sondern wurde \cite{tagging} entnommen. Die Kalibrationsfunktion für $\eta^{OS}$ lautet:
\begin{align}
\omega(\eta^{OS}) = p_1\left(\eta^{OS}-\left\langle \eta^{OS} \right\rangle\right) + p_0 .
\end{align}
$\left\langle \eta^{OS} \right\rangle$ steht dabei für das arithmetische Mittel der $\eta^{OS}$-Verteilung. Aus \cite{tagging} erhält man
\begin{align}
p_0 &= 0,382 \pm 0,003 \\
p_1 &= 0,981 \pm 0,024 \\
\left\langle \eta^{OS} \right\rangle &= 0,382
\end{align}
Hier wurde die statistischen Unsicherheiten angegeben, die man aus dem Fit der Kalibrationsfunktion erhält. Diese werden später im Fit durch gaußische Einschränkung der Parameter $p_0$ und $p_1$ berücksichtigt. Des Weiteren kommt es zu systematischen Unsicherheiten, die man unter anderem durch Vergleich diverser Zerfallskanäle abschätzt. Leider sind die systematischen Studien für die in dieser Analyse verwendete Kalibration bislang nicht abgeschlossen. In Kapitel \ref{kap:syst_tagging} wird der systematische Einfluss der Flavour Tagging Kalibration auf  $\SJPsi$ untersucht. Hierbei wird für eine vorläufige Abschätzung dann auf die systematischen Unsicherheiten der Kalibration aus 2011 \cite{lhcb-paper} zurückgegriffen.

Geladene Teilchen können je nach Ladung zum Teil sehr unterschiedlich mit dem Detektormaterial reagieren. Daher kommt es auch zu unterschiedlichen Rekonstruktionseffizienzen für \Bd\ und \Bdbar. Entsprechend müssen zwei Kalibrationsfunktionen 
\begin{align}
\omega^{\text{\Bd}}(\eta^{OS}) &= p_1(\text{\Bd})\left(\eta^{OS}-\left\langle \eta^{OS} \right\rangle\right) + p_0(\text{\Bd}) , \\
\omega^{\text{\Bdbar}}(\eta^{OS}) &= p_1(\text{\Bdbar})\left(\eta^{OS}-\left\langle \eta^{OS} \right\rangle\right) + p_0(\text{\Bdbar})
\end{align}
berücksichtigt werden. Für die Differenzen der Kalibrationsparameter liefert \cite{tagging}
\begin{align}
\Delta p_0 &= p_0(\text{\Bd}) - p_0(\text{\Bdbar}) = 0,0045 \pm 0,0053 \\
\Delta p_1 &= p_1(\text{\Bd}) - p_1(\text{\Bdbar}) = 0,001 \pm 0,05 .
\end{align}
Während $p_1$ für \Bd\ und \Bdbar\ sehr gut übereinstimmen, muss man das bei $p_0$ differenzierter betrachten. Auch hier ist man zwar im $1\sigma$-Bereich kompatibel, andere Studien der LHCb-Gruppe zeigen jedoch, dass die Tagging Asymmetrie $\Delta p_0$ berücksichtigt werden sollte, was auch hier geschieht. Dazu werden $\omega$ und $\Delta p_0$ so umdefiniert, dass
\begin{align}
\Delta p_0 &= \omega^{\text{\Bd}} - \omega^{\text{\Bdbar}}, \\
\omega^{\text{\Bd}} &= \omega + \frac{\Delta p_0}{2},  \\
\omega^{\text{\Bdbar}} &= \omega - \frac{\Delta p_0}{2}
\end{align}
gilt. Aufgrund der Fehlerwahrscheinlichkeit beim Tagging weicht die gemessene Eigenzeitverteilung von der tatsächlichen deutlich ab. Bei einem gemessenen \Bd\ ($d=1$) handelt es sich in $(1-\omega^{\text{\Bd}})\%$ der Fälle auch tatsächlich um ein \Bd\ ($d_{wahr}=1$), in $\omega^{\text{\Bdbar}}\%$ der Fälle jedoch um ein wahres \Bdbar\ ($d_{wahr}=-1$). Damit lautet die Wahrscheinlichkeitsdichtefunktion der gemessenen Verteilung
\begin{alignat}{3}
\nonumber\widetilde{\mathcal{P}}_{\text{gem.}}(t, d, \omega) &= && &&\delta_{d,1} \left[(1-\omega^{\text{\Bd}})\widetilde{\mathcal{P}}_{\text{wahr}}(t, d_{\text{wahr}}=1) + \omega^{\text{\Bdbar}}\widetilde{\mathcal{P}}_{\text{wahr}}(t, d_{\text{wahr}}=-1)\right]\\
\nonumber & && + &&\delta_{d,-1} \left[(1-\omega^{\text{\Bdbar}})\widetilde{\mathcal{P}}_{\text{wahr}}(t, d_{\text{wahr}}=-1) + \omega^{\text{\Bd}}\widetilde{\mathcal{P}}_{\text{wahr}}(t, d_{\text{wahr}}=1)\right] \\
\nonumber &= && &&\frac{1}{\mathcal{N}_t}\e^{-t/\tau} \left\lbrace 1-d\mu(1-2\omega)-d\Delta p_0 \right. \\
& && - && \left.\left[d(1-2\omega)-\mu(1-d\Delta p_0)\right]\SJPsi\sin(\sinarg)\right\rbrace. \label{eq:fit_pdf_vorlaeufig}
\end{alignat}

\subsubsection{Effektive Tagging Effizienz}
Zum Ende dieses Abschnittes soll nun noch die effektive Tagging Effizienz bestimmt werden. Dazu wird zunächst die Tagging Effizienz berechnet. Es wird jeweils ein Massenfit nach Kapitel \ref{kap:massenfit} mit allen Kandidaten und nur mit Kandidaten, denen ein Flavour zugeordnet werden konnte durchgeführt. Aus beiden Fits wird der Anteil des Signals bestimmt und man erhält:
\begin{align}
\epsilon_{\text{tag}} = (29,43\pm 0,85)\%.
\end{align}
Es bleibt die Bestimmung von $\mathcal{D}:=1-2\omega$. Dazu wird zunächst die $\eta^{\text{OS}}$-Verteilung der Daten auf $\omega$ umkalibriert. Im Anschluss wird das gewichtete Mittel von $1-2\omega$ berechnet. Als Gewichte dienen zur Extrahierung des Signals die sWeights, die man aus dem zuvor durchgeführten Massenfit berechnet. Dies führt zu:
\begin{align}
\mathcal{D} = 0,2474\pm 0,0095.
\end{align}
Die Fehlerrechnung für $\mathcal{D}$ wird in \cite{2010-analyse} beschrieben. Aus beiden Werten folgt dann die effektive Tagging Effizienz von
\begin{align}
\epsilon_{\text{eff}} = \epsilon_{\text{tag}} \mathcal{D}^2 = (1,80\pm 0,15)\%
\end{align}

\subsection{Eigenzeitauflösung und -akzeptanz}
Ein weiterer Effekt, der noch berücksichtigt werden muss, ist die endliche Ei\-gen\-zeit\-auf\-lösung des Detektors. Dies wird dadurch deutlich, dass auch Ereignisse mit negativer Eigenzeit rekonstruiert werden. Da diese unphysikalisch sind und u.a. auf Auflösungseffekte zurückzuführen sind, werden genau diese Ereignisse zur Bestimmung einer Auflösungsfunktion verwendet. Es handelt sich dabei zwar nur um eine Näherung der Eigenzeitauflösung, wie aber später gezeigt wird, ist das nicht relevant für das Endergebnis von $\SJPsi$ (siehe Kap. \ref{kap:aufloesung}). Wie in den Kapiteln \ref{kap:trigger} und \ref{kap:stripping} bereits erwähnt wurde, werden hierzu auf den Datensatz die High Level Trigger 2 Linie \texttt{Hlt2DiMuonJPsiDecision} sowie die Stripping Linie \texttt{BetaSBd2JPsiKsPrescaledLine} angewandt. Diese selektiert nicht auf die Eigenzeit des Teilchen, hat dafür weniger Statistik und wurde speziell für diesen Zweck eingerichtet.

In dieser Arbeit wird das Modell der mittleren Eigenzeitauflösung verwendet. Als Akzeptanzfunktion wird ein dreifacher Gauß der Form
\begin{align}
\mathcal{R}(t) = \sum_{i=1}^{3} \frac{f_i}{\sqrt{2\pi}\sigma_i}\e^{-\frac{t^2}{2\sigma_i^2}}
\end{align}
mit dem gemeinsamen Mittelwert $0$, den unterschiedlichen Breiten $\sigma_i$, sowie den relativen Anteilen $f_i$ der einzelnen Gauß-Funktionen gewählt. Dabei ist $\sum f_i = 1$ zu beachten. Somit erübrigt sich, $f_3$ als eigenständigen Parameter zu betrachten, es wird $f_3 = 1 - f_1 - f_2$ verwendet. Um Signal von Untergrund zu trennen, wird ein sFit angewandt. Da der Zerfallsvertex der hier behandelten \Bd-Mesonen hinreichend gut durch den $\JPsi$-Vertex festgelegt wird, wird die Analyse nur mit $\JPsi$ durchgeführt und deshalb zur Bestimmung der sWeights die rekonstruierte $\JPsi$-Masse herangezogen (siehe \cite{lhcb-paper}). Entgegen dem Massenfit der \Bd-Mesonen aus Gleichung (\ref{eq:pdf_masse}) wird hier als Wahrscheinlichkeitsdichtefunktion die Summe aus einer Gauß- und einer CrystalBall-Funktion verwendet. Die Crystallball-Funktion hat eine gaußförmige Basis, aber einen zu kleineren Werten als dem Mittelwert hin asymmetrischen, abgeflachten Teil, der den nicht detektierbaren Energieverlust durch Photonabstrahlung berücksichtigt. Sie ist durch
\begin{align}
\mathcal{CB}(m) = \frac{N}{\sigma\sqrt{2\pi}} \begin{cases} \exp(-\frac{(m-m_{\text{\Bd}})^2}{2\sigma^2}), & \text{für }\frac{m-m_{\text{\Bd}}}{\sigma}>-\alpha \\ \left(\frac{n}{|\alpha|}\right)^n \exp(-\frac{|\alpha|^2}{s}) \left(\frac{n}{|\alpha|}-|\alpha|-\frac{m-m_{\text{\Bd}}}{\sigma}\right)^{-n} & \text{für }\frac{m-m_{\text{\Bd}}}{\sigma}\leq -\alpha \end{cases} 
\end{align}
definiert. Der Parameter $\alpha$ beschreibt dabei den Übergang vom gaußartigen Teil in den auf einer Potenzfunktion besierenden abgeflachten Teil \cite{crystal_ball}. Abbildung \ref{fig:resolution} zeigt sowohl das Ergebnis des Massenfits als auch den Fit der Auflösungsfunktion. Die erhaltenen Parameter der Eigenzeitauflösung sind in Tabelle \ref{tab:resolution} aufgeführt.
\begin{figure}[hptb]
\centering
\includegraphics[width=\textwidth]{resolution}
\caption{Bestimmung der Auflösung: Die linke Hälfte zeigt den für die Bestimmung der sWeights durchgeführten Fit an die rekonstruierte $\JPsi$-Masse (oben) und die dazugehörigen Pulls (unten), die rechte Hälfte den Fit der Auflösungsfunktion (oben) und die entsprechenden Pulls (unten). Letztere scheinen nicht optimal zu sein, die Schwankungen sind recht groß, es wird jedoch in Kapitel \ref{kap:aufloesung} gezeigt, dass eine präzise Kenntnis der Auflösung nicht von Nöten ist.}
\label{fig:resolution}
\end{figure}
\begin{table}[hptb]
\centering
\caption{Ergebnisse des Fits der Eigenzeitauflösung}
\label{tab:resolution}
$\begin{array}{ll r@{\pm}l}
\hline 
\hline
\multicolumn{2}{l}{\text{Parameter}} & \multicolumn{2}{c}{\text{Wert}}\\
\hline
\sigma_{1}& [\ps] & 0,480 & 0,070 \\
\sigma_{2}& [\ps] & 0,04396 & 0,00094 \\
\sigma_{3}& [\ps] & 0,0932 & 0,0034 \\
f_{1} & & 0,00329 & 0,00099\\
f_{2} & & 0,739 & 0,027\\ \hline
\sigma_{\text{eff}}& [\ps] & 0,0665 & 0,0013 \\ \hline \hline
\end{array}$   
\end{table}
Die effektive Auflösung beträgt
\begin{align}
\sigma_{\text{eff}} = \sqrt{\sum_{i=1}^3 f_i \sigma_i^2} = (0,0665\pm 0,0013)\ps.
\end{align}
Im Fit wird die Auflösung dadurch berücksichtigt, dass die Wahrscheinlichkeitsdichtefunktion der Eigenzeitverteilung $\widetilde{\mathcal{P}}_{\text{gem.}}(t, \omega)$ aus Gleichung (\ref{eq:fit_pdf_vorlaeufig}) mit der Auflösungsfunktion $\mathcal{R}(t)$ gefaltet wird. 

Ein weiterer Punkt, der berücksichtigt werden muss, ist die Eigenzeitakzeptanz des Detektors. In der Analyse aus 2011 \cite{lhcb-paper} wurde gezeigt, dass hier keine großen Einflüsse erwartet werden, daher wird die Akzeptanzfunktion 
\begin{align}
\epsilon(t) = 1
\end{align}
gesetzt. Eine systematische Analyse und eine Abschätzung des Einflusses dieser Näherung findet sich in Kapitel \ref{kap:akzeptanz}.

\subsection{Fitfunktion}
Kombiniert man nun alle Effekte, die im vorigen Kapitel \ref{kap:eigenzeitverteilung}, aufgeführt und beschrieben wurden, so nimmt die Wahrscheinlichkeitsdichtefunktion für die Eigenzeitverteilung der \Bd-Mesonen die Form
\begin{alignat}{3}
\nonumber\mathcal{P}_{\text{gem.}}(t, d, \eta) &=\  &&  \epsilon(t)\left[\widetilde{\mathcal{P}}_{\text{gem.}}(t', d, \eta) \otimes \mathcal{R}(t-t')\right]\\
\nonumber &= &&\frac{1}{\mathcal{N}_t}\left[\e^{-t'/\tau} \left\lbrace 1-d\mu(1-2\omega)-d\Delta p_0 \right.\right.\\
& &&-\left.\vphantom{\frac{1}{\mathcal{N}_t}}\left.\left[d(1-2\omega)-\mu(1-d\Delta p_0)\right]\SJPsi\sin(\Delta m_d t')\right\rbrace\right]\otimes\mathcal{R}(t-t') \label{eq:fit_pdf}
\end{alignat}
an. Durch den Fit dieser Funktion erhält man eine Abschätzung für den \CP-Parameter $\SJPsi$. In Gleichung \ref{eq:fit_pdf} bezeichnen $t'$ die wahre Eigenzeit, $t$ die rekonstruierte Eigenzeit, $\tau$ die \Bd-Lebensdauer, $\Delta m_d$ die Oszillationsfrequenz des \Bd-Mesons, $\mu$ die Produktionsasymmetrie sowie $d$ den durch die Flavour-Tagging Algorithmen bestimmten Anfangszustand des \Bd. Dabei gilt für \Bd-Mesonen $d=1$, für \Bdbar\ $d=-1$. Die Mistagwahrscheinlichkeit des Flavour Taggings $\omega$ ist wiederum abhängig von der von den Algorithmen vorhergesagten Mistagwahrscheinlichkeit $\eta$ gemäß
\begin{align}
\omega(\eta) = p_1\left(\eta-\left\langle\eta\right\rangle\right) + p_0.
\end{align}

Die \CP-Asymmetrie wird ebenfalls durch die Eigenzeitauflösung sowie fehlerhaftes Flavour Tagging beeinflusst, hier gilt für die Messung \cite{lhcb-paper}
\begin{align}
\mathcal{A_{CP}}^{\text{meas}}(t) = (1-2\omega)\SJPsi \sin(\Delta m_d t') \otimes \mathcal{R}(t-t') \label{eq:cp_asymm_meas}
\end{align}


\section{Ergebnisse} \label{kap:fitergebnis}
Im Fit der Eigenzeitverteilung bleiben nicht alle Parameter frei. Fixiert werden zum einen die gesondert bestimmten Parameter der Eigenzeitauflösung (siehe Tab. \ref{tab:resolution}) und der Flavour-Tagging Kalibrationsparameter $\left\langle\eta\right\rangle = 0,382$, da dieser fehlerlos definiert ist\footnote{Etwaige Fehler sind in den Unsicherheiten von $p_0$ und $p_1$ enthalten.}. Des Weiteren werden einige Parameter gaußisch eingeschränkt, um deren statistische Unsicherheiten im Fit zu berücksichtigen. Dies sind die Produktionasymmetrie $\mu$ sowie die Kalibrationsparameter $p_0$, $p_1$ und $\Delta p_0$. Die verwendeten Werte sind aus \cite{lhcb-paper} für $\mu$ beziehungsweise \cite{tagging} für $p_0$, $p_1$ und $\Delta p_0$ entnommen und in Tabelle \ref{tab:constrained_parameters} aufgeführt.

\begin{table}[hptb]
\centering
\caption{Parameter, die im Fit entsprechend ihrer Unsicherheiten gaußisch eingeschränkt werden.}
\label{tab:constrained_parameters}
$\begin{array}{l r@{\pm}l }
\hline 
\hline
\text{Parameter} & \multicolumn{2}{c}{\text{Wert}}\\
\hline
p_0 & 0,382 & 0,003 \\
p_1 & 0,981 & 0,024 \\
\Delta p_0 & 0,0045 & 0,0053 \\
\mu & -0,015 & 0,013 \\ \hline \hline
\end{array}$ 
\end{table}
Als Parameter, die frei laufen, bleiben dementsprechend die \CP-Asymmetrie Amplitude $\SJPsi$, die Lebensdauer $\tau$ sowie die Oszillationsfrequenz $\Delta m_d$ übrig. Während der gesamten Analyse wurde der Parameter $\SJPsi$ verdeckt (Fachjargon: \glqq geblindet\grqq). Dabei wird das eigentliche Ergebnis um einen dem Experimentator unbekannten Wert verschoben. Diese Verschiebung wird mittels einer Zeichenkette berechnet. Dies soll verhindern, dass sich der Experimentator an älteren Messungen oder dem Weltmittelwert etc. orientiert und dahingehend seine Analyse beeinflusst. Erst nach Beendigung aller systematischen Studien (siehe Kapitel \ref{kap:systematik}) und beim Verfassen dieser Arbeit wurde die wahre Abschätzung von $\SJPsi$ aufgedeckt. Diese sei hier schon einmal vorweggenommen:
\begin{align}
\SJPsi = 0,711 \pm 0,063     \label{eq:fit_result}
\end{align}
Alle Resultate des Fits sind in Tabelle \ref{tab:fit_results} aufgeführt. Die gemessene Eigenzeitverteilung sowie die dazugehörigen Fitkurven sind in Abbildung \ref{fig:fit_result} in Schwarz (für \Bd) und in Rot (\Bdbar) dargestellt. Des Weiteren zeigt Abbildung \ref{fig:asymmetrie} die \CP-Asymmetrie des Signals.

\begin{table}[hptb]
\centering
\caption{Ergebnisse des Fits der Eigenzeitverteilung. Die angegebenen Unsicherheiten sind statistische.}
\label{tab:fit_results}
$\begin{array}{ll r@{\pm}l}
\hline 
\hline
\multicolumn{2}{l}{\text{Parameter}} & \multicolumn{2}{c}{\text{Wert}}\\
\hline
\SJPsi & & 0,711 & 0,063 \\
\tau & [\ps] & 1,498 & 0,017 \\
\Delta m_d & [\ps^{-1}] & 0,474 & 0,034 \\ \hline
p_0 & & 0,3815 & 0,0030\\
p_1 & & 0,977 & 0,024 \\
\Delta p_0 & & 0,0049 & 0,0050 \\
\mu & & -0,020 & 0,013 \\ \hline \hline
\end{array}$ 
\end{table}
\begin{figure}[hptb]
\centering
\includegraphics[width=\textwidth]{eigenzeitverteilung}
\caption{Ergebnis des Fits der Eigenzeitverteilung: Gemessene Eigenzeitverteilung der \Bd- (schwarz) bzw. \Bdbar-Mesonen (rot) mit entsprechendem Fitergebnis gemäß Gleichung (\ref{eq:fit_pdf}) und den Parametern aus Tabelle \ref{tab:fit_results} (oben) sowie dazugehörige Pull-Verteilung (unten).}
\label{fig:fit_result}
\end{figure}
\begin{figure}[hptb]
\centering
\includegraphics[width=\textwidth]{asymmetrie}
\caption{Darstellung der \CP-Asymmetrie $\mathcal{A_{CP}}$ nach Definition aus Gleichung \ref{eq:cp_asymm_def}. Für die dazugehörige Kurve wurden die aus dem Eigenzeitfit erhaltenen Parameter in Gleichung (\ref{eq:cp_asymm_meas}) eingesetzt. Das grüne Band entspricht den $1\sigma^{\text{stat.}}$ Abweichungen von $\SJPsi$ und $\Delta m_d$.}
\label{fig:asymmetrie}
\end{figure}