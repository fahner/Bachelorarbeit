\begin{titlepage}
\begin{center}
 
\Large\textbf{Fakultät für Physik und Astronomie\\
Ruprecht-Karls-Universität Heidelberg}

\vspace{15cm}

\normalsize
Bachelorarbeit in Physik\\
eingereicht von\\
\vspace{0.5cm}
\Large\textbf{Patrick Fahner}\\
\normalsize
\vspace{0.5cm}
geboren in Mannheim (Deutschland)\\
\vspace{0.5cm}
\Large\textbf{August 2013}

\newpage
\newpage

\normalsize
\boldmath
\Huge{\textbf{Messung von $\sin(2\beta)$ im Zerfall \Decaychannel}}
\unboldmath
\vfill
\normalsize
Diese Bachelorarbeit wurde von Patrick Fahner am\\
Physikalischen Institut der Universität Heidelberg\\
unter der Aufsicht von\\
Prof. Dr. Stephanie Hansmann-Menzemer \\
durchgeführt.
\end{center}
\end{titlepage}
\newpage
\hrule
\section*{\abstractname}
In dieser Arbeit wurde der CKM-Winkel $\sin(2\beta)$ bestimmt. Hierzu wurde der Zerfallskanal \Decaychannel\ ausgewertet in Daten, die 2012 am LHCb-Detektor bei einer Schwerpunktsenergie von $\sqrt{s}=8\tera\electronvolt$ aufgenommen wurden und einer integrierten Luminosität von ungefähr $2\femto\barn^{-1}$ entsprechen. Als Basis diente eine LHCb-Analyse der im Jahre 2011 aufgenommenen Daten (\cite{lhcb-paper}). Das Ergebnis 
\begin{align*}
\sin(2\beta) = 0,711 \pm 0,059(\text{stat.}) \pm 0,033(\text{syst.})
\end{align*}
ist sowohl mit dem Resultat der 2011-Analyse $\sin(2\beta) = 0.72 \pm 0,07 (\text{stat.}) \pm 0,04 (\text{syst.})$ (\cite{lhcb-paper}) als auch mit dem aktuellen Welt-Mittelwert von $\sin(2\beta) = 0,682 \pm 0,019$ (\cite{pdg-average}) kompatibel. Ein wichtiger Punkt dieser Arbeit war die Abschätzung systematischer Effekte. Dabei wurde deutlich, dass die systematische Unsicherheit im Wesentlichen von der Kalibration der Flavour Tagging Algorithmen bestimmt ist. \\
\hrule
\vfill
\hrule
\selectlanguage{english}
\hrule
\section*{\abstractname}
This thesis provides a measurement of the CKM-angle $\sin(2\beta)$. It evaluates the decay channel \Decaychannel\ of data taken by the LHCb experiment at a center of mass energy of $\sqrt{s}=8\tera\electronvolt$ with an integrated luminosity of about $2\femto\barn^{-1}$. Based on an LHCb analysis of 2011 one obtains
\begin{align*}
\sin(2\beta) = 0,711 \pm 0,059(\text{stat.}) \pm 0,033(\text{syst.})
\end{align*}
which is in good agreement with the 2011 LHCb result $\sin(2\beta) = 0.72 \pm 0,07 (\text{stat.}) \pm 0,04 (\text{syst.})$ (\cite{lhcb-paper}) as well as the world average $\sin(2\beta) = 0,682 \pm 0,019$ (\cite{pdg-average}). Furthermore this thesis focuses the study of systematic effects. The main contribution to the systematic error is the calibration of the flavour tagging algorithms. \\ \hrule


\selectlanguage{ngerman}



