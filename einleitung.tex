\chapter{Einleitung}
Das Standardmodell der Teilchenphysik beschreibt erfolgreich und mit hoher Präzision die bislang beobachteten Elementarteilchen sowie drei der vier elementaren Wechselwirkungen: die starke, elektromagnetische sowie die schwache Wechselwirkung. Nur die Gravitation kann nicht beschrieben werden. Trotz dieses Erfolgs gibt es offene Fragen, die das Standardmodell nicht beantworten kann: Was ist dunkle Materie? Wie kam es zur Asymmetrie von Teilchen und Antiteilchen (Baryogenese)? Solchen Fragen hat sich das \glqq Large Hadron Collider beauty (LHCb-)Experiment\grqq\ verschrieben. Die Antwort auf diese Fragen könnte in der Existenz neuer, bislang unentdeckter Teilchen liegen. LHCb ist auf der Suche nach etwaigen Hinweisen hierfür.  Um diese zu finden, ist es notwendig, das Standardmodell präzise zu vermessen. \cite{cern-courier, roadmap}

Die vorliegende Arbeit soll hierzu einen Beitrag leisten. Dazu wird der CKM-Winkel $\beta$ in der Form $\sin(2\beta)$ mit Hilfe der \CP-Asymmetrie des Zerfalls \Decaychannel gemessen. Die Zerfälle wurden im Jahre 2012 bei Proton-Proton-Kollisionen am Large Hadron Collider (LHC) des CERN in Genf bei einer Schwerpunktsenergie von $\sqrt{s}=8\tera\electronvolt$ aufgenommen.

Kapitel \ref{kap:experiment} bietet zunächst einen Überblick über das LHCb-Experiment und den Detektor selbst. Darauf folgt (Kapitel \ref{kap:cp-verletzung}) eine Beschreibung der verschiedenen Arten der \CP-Verletzung und wie sie sich im Zerfall \Decaychannel manifestiert, am Ende des Kapitels wird dann der Zusammenhang zum Standardmodell und dem CKM-Winkel $\beta$ hergestellt. Zur Messung von $\sin(2\beta)$ müssen die Daten sorgfältig ausgewählt werden. Die hierzu nötigen Schritte werden in Kapitel \ref{kap:datenselektion} beschrieben. Hiernach wird dann in Kapitel \ref{kap:analyse} die eigentliche Analyse beschrieben. Um diese zu komplettieren, enthält Kapitel \ref{kap:systematik} Systematische Studien, bevor dann noch einmal die wichtigsten Erkenntnisse zusammengefasst werden.