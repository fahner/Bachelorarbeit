\chapter{Einleitung}
Das Standardmodell der Teilchenphysik beschreibt erfolgreich und mit hoher Präzision die bislang beobachteten Elementarteilchen sowie drei der vier elementaren Wechselwirkungen: die starke, elektromagnetische sowie die schwache Wechselwirkung. Nur die Gravitation kann nicht beschrieben werden. Trotz dieses Erfolgs gibt es offene Fragen, die das Standardmodell nicht beantworten kann: Was ist dunkle Materie? Wie kam es zur Asymmetrie von Teilchen und Antiteilchen (Baryogenese)? Solchen Fragen hat sich das \glqq Large Hadron Collider beauty (LHCb) Experiment\grqq\ verschrieben. Die Antwort hierauf könnte in der Existenz neuer, bislang unentdeckter Teilchen liegen. LHCb ist auf der Suche nach etwaigen Hinweisen hierfür.  Um diese zu finden, ist es notwendig, das Standardmodell präzise zu vermessen. \cite{cern-courier, roadmap}

Die vorliegende Arbeit soll hierzu einen Beitrag leisten. Dazu wird der CKM\footnote{Abkürzung für die drei Physiker Cabibbo, Kobayashi und Maskawa, nach denen die CKM-Matrix, auch als Quark-Mischungsmatrix bekannt, benannt wurde.}-Winkel $\beta$ in der Form $\sin(2\beta)$ mit Hilfe der \CP-Asymmetrie des Zerfalls \Decaychannel\ gemessen. Die Zerfälle wurden im Jahre 2012 bei Pro\-ton-Prot\-on-Kol\-li\-si\-onen am Large Hadron Collider (LHC) des CERN\footnote{Conseil Européen pour la Recherche Nucléaire} in Genf bei einer Schwerpunktsenergie von $\sqrt{s}=8\tera\electronvolt$ aufgenommen. Der Zerfallskanal \Decaychannel\ wird gewählt, weil der Endzustand $\Ket{\JPsi\Kshort}$ ein fast reiner \CP-Eigenzustand ist und damit sowohl \Bd- als auch \Bdbar-Mesonen in diesen Zustand zerfallen können. Des Weiteren können \Bd\ und \Bdbar\ \glqq mischen\grqq\, d.h. sie können ineinander übergehen. Die \CP-Asymmetrie kommt nun dadurch zustande, dass es zu \CP-verletzenden Interferenzen von direktem Zerfall eines \Bd-/\Bdbar-Mesons und seinem Zerfall nach Mischung kommt.

Kapitel \ref{kap:experiment} bietet zunächst einen Überblick über das LHCb-Experiment und den Detektor selbst. Darauf folgt (Kapitel \ref{kap:cp-verletzung}) eine Beschreibung der verschiedenen Arten der \CP-Verletzung und wie sie sich im Zerfall \Decaychannel\ manifestieren, am Ende des Kapitels wird dann der Zusammenhang zum Standardmodell und dem CKM-Winkel $\beta$ hergestellt. Zur Messung von $\sin(2\beta)$ müssen die Daten sorgfältig ausgewählt werden. Die hierzu nötigen Schritte werden in Kapitel \ref{kap:datenselektion} beschrieben. Hiernach wird dann in Kapitel \ref{kap:analyse} die eigentliche Analyse beschrieben. Um diese zu komplettieren, enthält Kapitel \ref{kap:systematik} systematische Studien, bevor dann noch einmal die wichtigsten Erkenntnisse zusammengefasst werden.