\chapter{Datenselektion} \label{kap:datenselektion}
Dieses Kapitel beschreibt die notwendigen Schritte, um aus den Rohdaten des Detektors einen analysierbaren Datensatz (ein sog. NTupel) herzustellen. Wichtig ist dabei, den Datensatz von möglichst viel Untergrund zu bereinigen ohne sein eigentliches Signal zu verlieren.

\section{Bereitgestellter Datensatz}
Die in dieser Arbeit verwendeten Daten entstammen aus Proton-Proton-Kollisionen und wurden im Jahre 2012 vom LHCb-Detektor bei einer Schwerpunktsenergie von $\sqrt{s} = 8\tera\electronvolt$ aufgenommen. Die integrierte Luminosität beträgt ca. $2\femto\barn^{-1}$. Vom Betreuer wurde ein vorgefertigter Datensatz zur Verfügung gestellt. Wesentliche Schritte bei der Erstellung waren die Rekonstruktion der Ereignisse mittels der Software \textsc{Brunel} sowie der Analyse mit dem Programm \textsc{Da Vinci}. Dabei findet zur Reduzierung des Untergrunds eine Vorselektion statt, die Stripping genannt wird. Die Software selbst bietet für jeden Zerfallskanal entsprechende Sätze von Selektionskriterien an. Die hier Verwendeten werden in Kapitel \ref{kap:stripping} betrachtet.


\section{Selektionskriterien}
Wie bereits erwähnt, erfolgt die Reduzierung des Untergrunds in mehreren Schritten, die nun im Folgenden erläutert werden.

\subsection{Trigger} \label{kap:trigger}
Den erste Schritt bildet das Trigger-System, das schon während der Datennahme die Ereignisse sondiert. Der LHCb-Detektor verwendet dabei ein dreistufiges System: Der hardwarebasierte \glqq L0 Trigger \grqq reduziert die Ereignisrate von $40\mega\hertz$ auf $1\mega\hertz$. Im Anschluss folgt der zweiteilige, softwarebasierte \glqq High Level Trigger \grqq (HLT), der die Ereignisrate schlussendlich auf $2-3\kilo\hertz$ reduziert. \cite{trigger} 

Es stehen für verschiedenste Bedürfnisse diverse sogenannte \glqq Triggerlines\grqq\ zur Verfügung. Die in dieser Analyse verwendeten Triggerlines entsprechen denen der 2011 Analyse \cite{lhcb-paper} und wurden wie folgt gewählt:

\subsubsection{High Level Trigger 1 (HLT1)}
Hier wird der \texttt{HltDiMuonHighMass} Trigger gewählt. Dieser greift - wie der Name schon suggeriert - lediglich auf die Spuren der Myonen zurück, sodass nur das vom \Bd\ ausgesandte $J/\Psi \rightarrow \mu^+\mu^-$ für den Trigger verantwortlich ist. Es werden hierbei zur Selektion die Qualität des $J\Psi$-Vertex, die Myonen-Spuren, sowie die Masse und der (Transversal)Impuls des $J\Psi$ berücksichtigt. Die \texttt{HltDiMuonHighMassDecision} erzeugt kein Bias auf die Lebensdauer des \Bd-Mesons.

\subsubsection{High Level Trigger 2 (HLT2)}
In dieser Analyse werden zwei unterschiedliche Linien verwendet. Zur Bestimmung der Detektorauflösung wird die \texttt{Hlt2DiMuonJPsiDecision} verwendet, die ähnliche Variablen wie beim HLT1 verwendet und somit auch kein Bias erzeugt. Für die reguläre Analyse wird jedoch die \texttt{Hlt2DiMuonDetachedJPsiDecision} verwendet, die zusätzlich die Signifikanz der Zerfallszeit eines $J/\Psi$ berücksichtigt. Dadurch kommt es jedoch zu einem Bias der Lebensdauer. Der Vorteil dieser Triggerwahl liegt jedoch darin, dass mehr Statistik zur Verfügung steht.



\subsection{Stripping} \label{kap:stripping}
Bei der Erstellung des Datensatzes verwendete der Betreuer für das Stripping die Softwareversion Stripping20r0p1. Es wird nun überprüft, welche Selektionskriterien dabei verwendet wurden. Für die Auswahl des $\JPsi$ wurde auf den Kriteriensatz \texttt{Std\-Mass\-Constrained\-Jpsi\-2\-MuMu} zurückgegriffen, für das $\Kshort$ auf \texttt{StdLooseKsDD} sowie beim \Bd\ auf \texttt{Beta\-SBd2\-JpsiKs\-De\-tached\-Line} in der regulären Analyse bzw. \texttt{Beta\-SBd2\-JpsiKs\-Pre\-scaled\-Line} zur Bestimmung der Eigenzeitauflösung. Die in diesen Sätzen enthaltenen Kriterien sind in Tabelle \ref{tab:cuts_stripping} aufgeführt.

\begin{table}[hptb]
\centering
\caption{Im Stripping verwendete Kriterien zur Selektion von \Bd, $\JPsi$ und $\Kshort$}
\label{tab:cuts_stripping}
$\begin{array}{l|ll}
\hline \hline
\text{Zerfall} & \text{Variable} & \text{Wert} \\ \hline
$\Decaychannel$ & M($\Bd$) & \in [5150,5550] \mega\electronvolt/c^2 \\
& \frac{\chi^2_{vtx}}{\text{nDof}}($\Bd$) & < 10 \\ \hline
\JPsi \rightarrow \mu^+\mu^- & \frac{\chi^2_{track}}{\text{nDof}}(\mu^{\pm}) & < 3 \\
& \Delta \ln \mathcal{L}_{\mu\pi} & > 0 \\
& p_T(\mu^{\pm}) & > 500 \mega\electronvolt/c \\
& \frac{\chi^2_{vtx}}{\text{nDof}}(\JPsi) & < 16 \\
& |M(\mu^+\mu^-)-M(\JPsi)| & < 80 \mega\electronvolt/c^2 \\ \hline
\Kshort \rightarrow \pi^+\pi^- & p(\pi^{\pm}) & > 2000 \mega\electronvolt/c \\
& \frac{\chi^2_{vtx}}{\text{nDof}}(\Kshort) & < 20 \\
& \frac{\chi^2_{track}}{\text{nDof}}(\pi^{\pm}) & < 3 \\
& |M(\pi^+\pi^-)-M(\Kshort)| & < 64 \mega\electronvolt/c^2 \\
& \frac{\chi^2_{IP}}{\text{nDof}}(\pi^{\pm}) & > 4 \\ \hline \hline
\end{array}$
\end{table}

In der Tabelle bezeichnen $M$ die rekonstruierte Masse, $p$ den Impuls sowie $p_T$ den Transversalimpuls eines Teilchens. Zur Rekonstruktion werden weiterhin Spuren an die Detektortreffer gefittet. Um eine Aussage über die Güte des Fits zu erhalten, betrachtet man hier das entsprechende auf die Zahl der Freiheitsgrade (nDoF) normierte $\chi_{track}^2$. Analog gilt dies für die Rekonstruktion der Vertices ($\chi_{track}^2$). Je näher das reduzierte $\chi^2$ der 1 kommt, desto besser ist der Fit. IP steht für den Stoßparameter und $Delta \ln \mathcal{L}_{\mu\pi}$ ist ein Maß für die Wahrscheinlichkeit, ein Myon als Pion zu interpretieren und umgekehrt. Ist der Wert größer als Null, so ist die Wahrscheinlichkeit, dass das Teilchen ein Myon ist größer, als diejenige, ein Pion zu sein.

\subsection{Verwendete Spurklassen} \label{kap:downstream}
Für die Rekonstruktion der $J/\Psi$ werden ausschließlich \glqq Long\grqq-Spuren verwendet. Diese passieren das gesamte Rekonstruktionssystem. Durch die relativ lange Lebensdauer des $\Kshort$ kommt es in etwa 2/3 der Fälle vor, dass der VELO dieses nicht mehr registriert. Hinterlassen Teilchen nur in den TT und T Stationen Spuren, so spricht man von \glqq Downstream\grqq-Spuren (siehe auch Kap. \ref{kap:spurklassen}. Diese Analyse beschränkt sich auf ebenjene. Damit hat man im Vergleich zu $\Kshort$ aus Long-Spuren mehr Statistik zur Verfügung, muss aber bei Qualität der Rekonstruktion Einbußen hinnehmen, da die Informationen des VeLo fehlen. Insbesondere leidet die Präzision der Impuls- und Positionsmessungen. Folglich dürfen die Selektionskriterien bei Downstream-Spuren teilweise nicht so hart sein wie bei Long-Spuren. \cite{lhcb-paper}

\subsection{Zusätzliche Selektionskriterien}
Um den Datensatz noch besser vom Untergrund zu bereinigen, werden einige Kriterien aus dem Stripping verschärft und weitere eingeführt (siehe Tab. \ref{tab:cuts_offline}). Diese wurden aus \cite{lhcb-paper} übernommen.

\begin{table}[hptb]
\centering
\caption{Zusätzlich eingeführte Kriterien zur Untergrundbereinigung bzw. Selektion von \Bd, $\JPsi$ und $\Kshort$}
\label{tab:cuts_offline}
$\begin{array}{l|ll}
\hline \hline
\text{Zerfall} & \text{Variable} & \text{Wert} \\ \hline
$\Decaychannel$ & M($\Bd$) & \in [5170,5420] \mega\electronvolt/c^2 \\
& \tau($\Bd$) & > 0,3\ps \\
& \sigma_\tau($\Bd$) & < 0,2\ps \\
& \frac{\chi^2_{DTF(B+PV)}}{\text{nDof}}($\Bd$) & < 5 \\
& \frac{\chi^2_{IP}}{\text{nDof}}($\Bd$) & < 20 \\ 
& \frac{\chi^2_{IP}}{\text{nDof}}($\Bd$) \text{ des nächstbesten PV} & > 50 \\ \hline
\JPsi \rightarrow \mu^+\mu^- & \frac{\chi^2_{vtx}}{\text{nDof}}(\JPsi) & < 11 \\
& |M(\mu^+\mu^-)-M(\JPsi)| & \in [3030,3165] \mega\electronvolt/c^2 \\ \hline
\Kshort \rightarrow \pi^+\pi^- & \frac{\tau}{\sigma_\tau}(\Kshort) & > 5 \\
& \frac{l}{\sigma_l}(\Kshort) & > 5 \\
& |M(\pi^+\pi^-)-M(\Kshort)| & \in [475,520] \mega\electronvolt/c^2 \\ \hline \hline
\end{array}$
\end{table}
Die neu eingeführten Größen sind hier die Eigenzeit $\tau$ und die Flugstrecke $l$ sowie deren Unsicherheit $\sigma_\tau$ und $\sigma_l$. Weiterhin gibt es noch einen kinematischen Fit des Zerfallsbaums (\glqq DecayTreeFit\grqq - DTF). Um die Wirkung der einzelnen Kriterien zu untersuchen, werden alle bis auf das zu untersuchende Kriterium angewandt und in der Massenverteilung das Signal-zu-Untergrund-Verhältnis $S/B$ bestimmt. Als Parametrisierung für den Fit der Massenverteilung wird ein doppelter Gauß verwendet, der Untergrund wird durch eine Exponentialfunktion modelliert (mehr dazu in Kap. \ref{kap:massenfit}). Zur Berechnung des Signals und des Untergrundes werden der Doppelgauß bzw. die Exponentialfunktion im Bereich von $\pm 3\sigma$ um den Mittelwert ausgewertet. Des Weiteren ist es von Interesse wie viel Signal und wie viel Untergrund man durch das entsprechende Kriterium verliert. Dazu werden die Verhältnisse $\epsilon_{\text{sig}}$ ($\epsilon_{\text{bkg}}$) von Signal (Untergrund) bei Anwendung aller Kriterien zu Signal (Untergrund) ohne ebenjenes Kriterium berechnet. Die Ergebnisse dieser Berechnungen sind in Tabelle \ref{tab:cuts_efficiency} aufgeführt.

\begin{table}[hptb]
\centering
\caption{Berechnung des Signal-zu-Untergrund-Verhältnisses $S/B$ sowie der Effizienzen $\epsilon_{\text{sig}}$ für Signal und $\epsilon_{\text{bkg}}$ für Untergrund}
\label{tab:cuts_efficiency}
$\begin{array}{l|ccc}
\hline \hline
\text{ausgelassenes Kriterium} & S/B & \epsilon_{\text{sig}} & \epsilon_{\text{bkg}}\\ \hline
M($\Bd$) \in [5170,5420] \mega\electronvolt/c^2 & 4,24 & 1,000 & 1,000 \\
\tau($\Bd$) > 0,3\ps & 2,71 & 0,955 & 0,610\\
\sigma_\tau($\Bd$) < 0,2\ps & 4,24 & 1,000 & 1,000 \\
\frac{\chi^2_{DTF(B+PV)}}{\text{nDof}}($\Bd$) < 5 & 3,58 & 0,984 & 0,831 \\
\frac{\chi^2_{IP}}{\text{nDof}}($\Bd$) < 20 & 3,67 & 0,992 & 0,860 \\ 
\frac{\chi^2_{IP}}{\text{nDof}}($\Bd$) \text{ des nächstbesten PV} > 50 & 3,50 & 0,979 & 0,809 \\ 
\frac{\chi^2_{vtx}}{\text{nDof}}(\JPsi) < 11 & 4,19 & 0,995 & 0,982\\
|M(\mu^+\mu^-)-M(\JPsi)| \in [3030,3165] \mega\electronvolt/c^2 & 4,05 & 0,997 & 0,953\\ 
\frac{\tau}{\sigma_\tau}(\Kshort) > 5 & 4,18 & 0,995 & 0,982\\
\frac{l}{\sigma_l}(\Kshort) > 5 & 4,24 & 1,000 & 1,000 \\
|M(\pi^+\pi^-)-M(\Kshort)| \in [475,520] \mega\electronvolt/c^2 & 3,37 & 0,985 & 0,782\\ \hline
\text{alle angewandt} & 4,24 & 1,000 & 1,000 \\ \hline \hline
\end{array}$
\end{table}
Am Beispiel der Eigenzeit $\tau(\text{\Bd})$ soll noch einmal deutlich gemacht werden, wie die Tabelle zu lesen ist: Im Vergleich zur Anwendung aller Kriterien ($S/B=4,24$) ist das Signal-zu-Untergrund-Verhältnis ohne die Selektion anhand der Eigenzeit mit 2,71 deutlich schlechter. Während dieses Kriterium von 95,5\% des Signals passiert wird, ist dies nur bei 61,0\% des Untergrunds der Fall. Damit wird man fast 40\% des Untergrunds los, ohne viel Signal wegzuwerfen. Anhand der Werte in Tabelle \ref{tab:cuts_efficiency} sieht man, dass dieses Kriterium das effektivste ist.

\subsubsection{Bester Kandidat}
Es ist äußerst unwahrscheinlich, dass es mehrere \Decaychannel-Zerfälle in einem Ereignis gibt. Jedoch kann es vorkommen, dass es mehr als ein rekonstruiertes \Bd im Ereignis gibt. Da aber nur ein \Bd am Zerfall beteiligt ist, wird der beste Kandidat anhand des kleinsten $\chi^2_{DTF}/\text{nDoF}$ des DecayTreeFit identifiziert. \cite{lhcb-paper}

\subsubsection{Phantome}
Für die Daten aus 2012 gibt es ein neues Kriterium, das in der Analyse der 2011 Daten (\cite{lhcb-paper}) noch nicht zur Verfügung stand und deshalb hier gesondert betrachtet wird: Die Analysesoftware gibt nun für die Pionen- und Myonenspuren eine Wahrscheinlichkeit an, dass es sich bei dieser Spur nur um ein Phantom handelt. Bei der Suche nach einer Schranke für dieses Kriterium wurde intuitiv 0,5 gewählt. Denn ist die Wahrscheinlichkeit kleiner als 0,5, dann ist es wahrscheinlicher, dass es sich auch wirklich um eine Spur handelt als dass es ein Phantom ist.

Wendet man dieses Kriterium auf Myonen an, so erhält man die Effizienzen $\epsilon_{\text{sig}}=0,999$ und $\epsilon_{\text{bkg}}=0.978$. Damit ist es bei Weitem nicht so effektiv wie beispielsweise die Eigenzeitselektion, leistet aber dennoch einen Beitrag zur Bereinigung des Datensatzes.

Im Falle von Pionen kommt es zu Problemen. Eine Schranke bei 0,5 führt hier zu $\epsilon_{\text{sig}}=0,660$ und $\epsilon_{\text{bkg}}=0.561$. Leider geht deutlich zu viel Signal verloren. Es hat sich zudem herausgestellt, dass bei \glqq Downstream-Pionen\grqq die Wahrscheinlichkeitsberechnung in der Analysesoftware nicht korrekt kalibriert wurde und notwendige Korrekturen bei Erstellung dieses Datensatzes nicht berücksichtigt wurden. Daher sind die zur Verfügung gestellten Werte nicht aussagekräftig und auf eine Selektion mittels Phantom-Wahrscheinlichkeit der Pionspuren wird verzichtet.

\subsubsection{Fitbereiche}
In den Analysen werden beim Fit die Massenbereiche zusätzlich eingeschränkt. Bei der Bestimmung der Detektorauflösung werden $\JPsi$ im Bereich $[3035, 3160]\mega\electronvolt/c^2$ betrachtet, im regulären Fit wird nur \Bd-Kandidaten im Bereich $[5230, 5330]\mega\electronvolt/c^2$ berücksichtigt.

\section{Verwendeter Datensatz}
Nach Anwendung aller Selektionskriterien stehen insgesamt 62184 Signalkandidaten zur Verfügung. Für diese Analyse ist essentiell, dass der Flavour der Mesonen (\Bd, \Bdbar) zum Zeitpunkt t=0 bekannt ist. Hier gibt es ebenfalls entsprechende Algorithmen, die manchmal einen Anfangsflavour zuordnen können, manchmal aber auch nicht (siehe Kap. \ref{kap:tagging}). Zerfälle ohne diese Zuordnung sind für diese Analyse nutzlos, daher stehen tatsächlich nur insgesamt 20109 Signalkandidaten zur Verfügung.