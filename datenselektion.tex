\chapter{Datenselektion}
\section{Bereitgestellter Datensatz}
\section{Schnitte}
Um Signal besser vom Untergrund zu trennen, werden in mehreren Schritten diverse Schnitte angewandt.
\subsection{Trigger} \label{kap:trigger}
Den erste Schritt bildet das Trigger-System, das schon während der Datennahme die Ereignisse sondiert. Der LHCb-Detektor verwendet dabei ein dreistufiges System: Der hardwarebasierte \glqq L0 Trigger \grqq reduziert die Ereignisrate von $40\mega\hertz$ auf $1\mega\hertz$. Im Anschluss folgt der zweiteilige, softwarebasierte \glqq High Level Trigger \grqq (HLT), der die Ereignisrate schlussendlich auf $2\kilo\hertz$ reduziert.\cite{trigger} 

Die in dieser Analyse verwendeten Trigger-Entscheidungen entsprechen denen der 2011 Analyse \cite{lhcb-paper} und wurden wie folgt gewählt:

\subsubsection{L0 Trigger}
Hier wird keine spezielle Entscheidung benötigt.

\subsubsection{High Level Trigger 1 (HLT1)}
Hier wird die \texttt{HltDiMuonHighMassDecision} gewählt. Diese greift - wie der Name schon suggeriert - lediglich auf die Spuren der Myonen zurück, sodass nur das vom \Bd ausgesandte $J/\Psi \rightarrow \mu^+\mu^-$ für den Trigger verantwortlich ist. Es werden hierbei Schnitte auf die Qualität des $J\Psi$-Vertex, die Myonen-Spuren, sowie die Masse und den (Transversal)Impuls des $J\Psi$ angewandt. Die \texttt{HltDiMuonHighMassDecision} erzeugt kein Bias auf die Lebensdauer des \Bd-Mesons.

\subsubsection{High Level Trigger 2 (HLT2)}
In dieser Analyse werden zwei unterschiedliche Entscheidungen verwendet. Zur Bestimmung der Detektorauflösung wird die \texttt{Hlt2DiMuonJPsiDecision} verwendet, die ähnliche Variablen wie beim HLT1 verwendet und somit auch kein Bias erzeugt. Für die reguläre Analyse wird jedoch die \texttt{Hlt2DiMuonDetachedJPsiDecision} verwendet, die zusätzlich die Signifikanz der Zerfallszeit eines $J/\Psi$ berücksichtigt. Dadurch kommt es jedoch zu einem Bias der Lebensdauer. Der Vorteil dieser Triggerwahl liegt jedoch darin, dass mehr Statistik zur Verfügung steht.

\subsection{Downstream Spuren}
Für die Rekonstruktion der $J/\Psi$ werden ausschließlich sog. \glqq Long\grqq-Spuren verwendet. Diese passieren das gesamte Rekonstruktionssystem. Durch die relativ lange Lebensdauer des $\Kshort$ kommt es in etwa 2/3 der Fälle vor, dass der VeLo dieses nicht mehr registriert. Hinterlassen Teilchen nur in den TT und T Stationen Spuren, so spricht man von \glqq Downstream\grqq-Spuren. Diese Analyse beschränkt sich auf ebenjene. Damit hat man im Vergleich zu $\Kshort$ aus Long-Spuren mehr Statistik zur Verfügung, muss aber bei Qualität der Rekonstruktion Einbußen hinnehmen, da die Informationen des VeLo fehlen. Insbesondere leidet die Präzision der Impuls- und Positionsmessungen. Folglich dürfen die Schnitte bei Downstream-Spuren teilweise nicht so hart sein wie bei Long-Spuren. \cite{lhcp-paper}

\subsection{Stripping} \label{kap:stripping}
!!! Achtung !!! Anpassen !!! Welches Stripping wurde verwendet???

Die Schnitte, die hierbei angewandt wurden, sind in Tabelle \ref{tab:cuts_stripping} aufgeführt.

\begin{table}[hptb]
\centering
\caption{Im Stripping verwendete Schnitte zur Selektion von \Bd, $\JPsi$ und $\Kshort$}
\label{tab:cuts_stripping}
$\begin{array}{l|ll}
\hline \hline
\text{Zerfall} & \text{Variable} & \text{Wert} \\ \hline
$\Decaychannel$ & M($\Bd$) & \in [5150,5550] \mega\electronvolt/c^2 \\
& \frac{\chi^2_{vtx}}{\text{nDof}}($\Bd$) & < 10 \\ \hline
\JPsi \rightarrow \mu^+\mu^- & \frac{\chi^2_{track}}{\text{nDof}}(\mu^{\pm}) & < 3 \\
& \Delta \ln \mathcal{L}_{\mu\pi} & > 0 \\
& p_T(\mu^{\pm}) & > 500 \mega\electronvolt/c \\
& \frac{\chi^2_{vtx}}{\text{nDof}}(\JPsi) & < 16 \\
& |M(\mu^+\mu^-)-M(\JPsi)| & < 80 \mega\electronvolt/c^2 \\ \hline
\Kshort \rightarrow \pi^+\pi^- & p(\pi^{\pm}) & > 2000 \mega\electronvolt/c \\
& \frac{\chi^2_{vtx}}{\text{nDof}}(\Kshort) & < 20 \\
& \frac{\chi^2_{track}}{\text{nDof}}(\pi^{\pm}) & < 3 \\
& |M(\pi^+\pi^-)-M(\Kshort)| & < 64 \mega\electronvolt/c^2 \\
& \frac{\chi^2_{IP}}{\text{nDof}}(\pi^{\pm}) & > 4 \\ \hline \hline
\end{array}$
\end{table}

Hierbei bezeichnen $M$ die rekonstruierte Masse, $p$ den Impuls sowie $p_T$ den Transversalimpuls eines Teilchens. Zur Rekonstruktion werden Spuren an die Detektortreffer gefittet. Um eine Aussage über die Güte des Fits zu erhalten, betrachtet man hier das entsprechende auf die Zahl der Freiheitsgrade (nDoF) normierte $\chi_{track}^2$. Analog gilt dies für die Rekonstruktion der Vertices ($\chi_{track}^2$). Je näher das reduzierte $\chi^2$ der 1 kommt, desto besser ist der Fit. !!! Impact Parameter !!! $Delta \ln \mathcal{L}_{\mu\pi}$ ist ein Maß für die Wahrscheinlichkeit, ein Myon als Pion zu interpretieren.

\subsection{Zusätzliche Schnitte}
Um den Datensatz noch besser vom Untergrund zu bereinigen, werden einige Schnitte aus den Stripping verschärft und weitere eingeführt (siehe Tab. \ref{tab:cuts_offline}). Diese wurden aus \cite{lhcb-paper} übernommen.

\begin{table}[hptb]
\centering
\caption{Zusätzlich eingeführte Schnitte zur Untergrundbereinigung bzw. Selektion von \Bd, $\JPsi$ und $\Kshort$}
\label{tab:cuts_offline}
$\begin{array}{l|ll}
\hline \hline
\text{Zerfall} & \text{Variable} & \text{Wert} \\ \hline
$\Decaychannel$ & M($\Bd$) & \in [5170,5420] \mega\electronvolt/c^2 \\
& \tau($\Bd$) & > 0,3\ps \\
& \sigma_\tau($\Bd$) & < 0,2\ps \\
& \frac{\chi^2_{DTF(B+PV)}}{\text{nDof}}($\Bd$) & < 5 \\
& \frac{\chi^2_{IP}}{\text{nDof}}($\Bd$) & < 20 \\ 
& \frac{\chi^2_{IP}}{\text{nDof}}($\Bd$) \text{ des nächstbesten PV} & > 50 \\ \hline
\JPsi \rightarrow \mu^+\mu^- & \frac{\chi^2_{vtx}}{\text{nDof}}(\JPsi) & < 11 \\
& |M(\mu^+\mu^-)-M(\JPsi)| & \in [3030,3165] \mega\electronvolt/c^2 \\ \hline
\Kshort \rightarrow \pi^+\pi^- & \frac{\tau}{\sigma_\tau}(\Kshort) & > 5 \\
& \frac{x}{\sigma_x}(\Kshort) & > 5 \\
& \frac{\chi^2_{track}}{\text{nDof}}(\pi^{\pm}) & < 3 \\
& |M(\pi^+\pi^-)-M(\Kshort)| & \in [475,520] \mega\electronvolt/c^2 \\ \hline \hline
\end{array}$
\end{table}
Die neu eingeführten Größen sind hier die Zerfallszeit $\tau$ und die Flugstrecke $x$ sowie deren Unsicherheit $\sigma_\tau$ und $\sigma_x$. Weiterhin gibt es noch einen kinematischen Fit des Zerfallsbaums (\glqq DecayTreeFit\grqq - DTF). Um die Wirkung der einzelnen Schnitte zu untersuchen, werden alle Schnitte bis auf den zu untersuchenden angewandt und in der Massenverteilung das Signal-zu-Untergrund-Verhältnis bestimmt. Dieses wird dann mit den entsprechenden Werten bei Anwendung aller Schnitte verglichen.

!!! Muss fortgesetzt werden !!!


\subsection{Geister-Wahrscheinlichkeit}

!!! Hier auch !!!

\subsection{Bester Kandidat}
Es ist äußerst unwahrscheinlich, dass es mehrere \Decaychannel-Zerfälle in einem Ereignis gibt. Jedoch kann es vorkommen, dass es mehr als ein rekonstruiertes \Bd im Ereignis gibt. Da aber nur ein \Bd am Zerfall beteiligt ist, wird der beste Kandidat anhand des kleinsten $\chi^2_{DTF}/\text{nDoF}$ des DecayTreeFit identifiziert. \cite{lhcb-paper}

\subsection{Fitbereiche}
In den Analysen werden beim Fit die Massenbereiche zusätzlich eingeschränkt. Bei der Bestimmung der Detektorauflösung werden $\JPsi$ im Bereich $[3035, 3160]\mega\electronvolt/c^2$ betrachtet, im regulären Fit wird nur \Bd-Kandidaten im Bereich $[5230, 5330]\mega\electronvolt/c^2$ berücksichtigt.

\section{Verwendeter Datensatz}
